\section{Introduction}
In light of the escalating advancements in photo-realistic rendering technologies, a surge of academic interest has emerged in the fields of 3D scene acquisition, representation, and re-rendering. The seamless integration of these technologies has enabled novel applications, including live 3D streaming, real-time 3D conferencing, holographic media, augmented reality, and the metaverse, poised to revolutionize the digital landscape in the near future.
Photo-realistic rendering platforms facilitate the rendering of dynamic 3D scenes or novel viewpoints, achieved through the concurrent and precise capture of multi-view image sequences using synchronized multi-camera devices. 


1. Most of previous camera array systems adopt simple uniform sampling strategy to densely capture 3D scene, i.e., arrange cameras in a plane or hemisphere grid.
2. For the 3D scene contains occlusion and known geometry, e.g., city with many building, rooms, flat, etc. To comprehensively capture 3D information in a given scene, we need to carefully optimize camera's position and pose to cover the all the information in 3D space and reduce the hardware resource allocation, i.e., number of camera.
3. In this paper, we solve camera layout design problem that optimizes the camera pose and position to maximize the system return with a cardinality constraint, e.g., number of camera. The naive solution is discrete the valid camera pose and position, and then exhaustively search all valid configuration of camera layout, and select the maximal layout, which yields a NP-hard problems.
4. To solve the camera layout problem in a practical polynomial time manner, we observe the diminishing return property of camera layout, that is incrementally increase camera quantity in a camera layout configuration, the additional system return starts to decrease.
5. We adopt submodular optimization as weapon to solve the camera layout optimization and explore the mathematical optimal camera layout configuration. We first build a graph to represent camera arrangement configuration, each vertex is camera pose and position, and edge that connect two adjacent cameras, the weight for edge is calculated by metrics (e.g., semantic similarity, mutual coverage) of two views. We formulate camera pose and position arrangement problem as the graph selection and clustering via submodular optimization, to maximize the entropy rate of random walk in a graph.
%7. camera design problem contains the nature of diminishing return property.
%maximizes the camera maximizes the camera observation coverage and minimize the number of camera usage in the constraint of 3D occlusion and space limitation, so that optimize the quality of reconstructed 3D space. 

\subsection{introduction}
Previous camera array systems have primarily adopted uniform sampling strategies, with cameras uniformly distributed in either plane or hemisphere formations for dense 3D scene capture (1). However, when dealing with scenarios involving occlusions and well-defined geometry, such as urban landscapes with a multitude of buildings, rooms, and dwellings, optimizing camera positioning and orientation becomes crucial for capturing a comprehensive 3D dataset while minimizing hardware demands, particularly the number of cameras.

This research endeavors to address the camera layout design problem by optimizing pose and position to maximize system performance under a cardinality constraint, like the total count of cameras. The straightforward approach, though, would be to exhaustively search all possible combinations of discrete camera positions and orientations – a computationally NP-hard task (3).

We recognize the diminishing return nature of camera layouts; adding more cameras incrementally leads to a diminishing increase in system return. To tackle this problem efficiently in practical settings, we leverage submodular optimization as a powerful tool for finding near-optimal solutions within a polynomial time complexity.

To implement this, we construct a graph representation, where vertices denote individual camera placements and orientations, and edges connect adjacent cameras. Edge weights are determined by factors like semantic similarity and mutual view coverage between the corresponding views. We mathematically formulate the camera pose and position arrangement problem as a graph selection and clustering problem, utilizing submodular optimization to maximize the entropy rate of a random walk on the graph.

In doing so, we sidestep the exhaustive search by seeking a sub-optimal but computationally efficient solution that captures the diminishing returns in camera layout configuration. This approach permits us to strike a balance between coverage and resource optimization, ultimately contributing to improved 3D reconstructions in scenarios with occlusions and complex geometries.


\subsection{Others}

Previous camera array systems have predominantly relied on a straightforward uniform sampling strategy, with cameras deployed in structured formations like planar or hemispherical grids for dense 3D scene capture (1). However, in scenarios where multi-level occlusion and intricate geometry exist, such as urban landscapes with numerous buildings, rooms, and apartment complexes, an optimized camera configuration is vital for capturing comprehensive 3D data efficiently. This requires balancing the need to cover the full 3D space with minimizing hardware resources, particularly the number of cameras employed.

In this work, we delve into the intricate challenge of designing camera arrays that optimally address the dual objectives of maximizing view overlap and minimizing camera redundancy, while respecting constraints imposed by three-dimensional occlusions and spatial limitations (3). To tackle this problem, we propose a novel approach that utilizes graph theory.

By constructing a graph representation, we capture the relationship between each camera’s pose and position, where vertices are designated for individual camera configurations. Edges connect neighboring cameras, and their weights are determined by the mutual benefit between their respective viewpoints, as quantified by metrics like semantic similarity and mutual view coverage. We mathematically formulate the problem of arranging camera poses and positions as a graph selection and clustering task, leveraging the principles of submodular optimization.

Submodular optimization principles enable us to maximize the efficiency of data extraction from the graph by optimizing the entropy rate of a random walk. This not only ensures that every node in the graph contributes meaningfully to the overall coverage but also minimizes the number of cameras necessary to achieve this, ultimately enhancing the quality and resource utilization of the reconstructed 3D space. The resulting method embodies a novel, computational-driven strategy that outperforms traditional uniform sampling techniques in scenes with complex geometries and occlusions.




Modern real-time 3D acquisition systems, particularly those relying on multiple area array cameras or light field cameras, demand an extensive array of camera resources, to ensure the capture of an extensive views of scene with minimal occlusion for further 3D computation.
The redundancy of view point is necessary for enhancing 3D reconstruction accuracy and robustness, yet it also underscores the computational and resource-intensive nature of these systems that require efficient optimization strategies for real-time performance.
%These systems then rely on high-performance computing resources in the software domain to develop explicit or implicit 3D representations. 
%The process involves computationally intensive operations to optimize the unknown parameters of these models, with the ultimate aim of storing, transmitting, and rendering intricate details, including geometry, texture, lighting, and radiometry, contributing to the immersive experiences these technologies provide.


%While numerous past research efforts in enhancing the overall performance of 3D acquisition and representation systems have adhered to a “more is better” mindset, where, on the hardware front, increased camera resources are thought to facilitate denser viewpoint sampling, and in the software realm, larger parameter sets are employed for more intricate model expressions. Concomitantly, the system’s soft and hardware configuration and deployment may exhibit “less restraints” in certain circumstances. Firstly, redundant resource allocation exists, indicating that judiciously reducing resources does not significantly degrade the system’s performance. Secondly, it is important to note that simply scaling up the hardware and software resources does not lead to unlimited performance gains, as there is a plateau. Thirdly, optimizing critical parameters has a pronounced impact on enhancing system performance. We observe that the 3D acquisition and expression systems exhibit a characteristic diminishing marginal returns: as more resources are invested, the incremental performance gains gradually decline, and targeted optimization of key configuration parameters can trigger substantial leaps in overall performance.

While numerous past research efforts have predominantly focused on increasing in hardware resources (e.g., more cameras) and software parameters, a more nuanced understanding is emerging. These efforts have indicated that while more resources initially contribute to greater precision, the system’s performance improvement plateaus beyond a certain point due to redundancy and resource constraints. This suggests that optimizing key parameters holds critical importance, as it can lead to significant leaps in system efficiency, despite the diminishing returns trend.


However, the existing theoretical framework lacks comprehensive support for optimizing the intricate mathematical link between discrete hardware-software configuration parameters in 3D information processing systems. The lack of methodological guidance hampers the development of effective loss functions and optimization strategies, which are fundamental for designing optimal systems. Discrete parameter optimizations, often NP-hard, call for exhaustive search methods, making them impractical in real-world scenarios.


Consequently, the ambitions of achieving resource efficiency and targeted performance enhancements, epitomized by “cost-effective” and “targeted resource allocation,” pose challenges in the practical adoption of 3D-enhanced applications, such as holographic projections and virtual reality, to adaptable and scalable platforms. This lack of tailored optimization prevents smooth integration and systematic improvements in performance, leaving the full potential of these systems largely unexplored.


To address this issue, adopting a system-centric approach to model and optimize discrete parameters in 3D acquisition and expression is of paramount significance. Establishing a mathematical model between discrete parameters and system performance offers a foundation for quantitative design and optimization, enabling systematic efficiency improvements across soft and hardware platforms and facilitating efficient configuration under resource constraints.


In the context of diminishing returns, the inherent submodularity can be strategically leveraged to optimize these systems, using polynomial-time algorithms that can navigate intractable optimization challenges associated with NP-hard problems. These algorithms yield superior approximation techniques, conducive to optimizing multiple discrete parameters simultaneously.


Lastly, the discrete nature of system parameter combinations aligns closely with practical challenges encountered in industry and everyday applications, particularly in large-scale systems with complex parameters lacking clear mathematical relationships. Optimizing these discrete parameters directly allows for precise, theoretical calculations of limits and optimal configurations, delivering superior design outcomes for large-scale projects, thereby highlighting the pivotal role of discrete parameter optimization in unlocking untapped potential.

%With the increasing prevalence of high-speed, low-latency internet connectivity and photo-realistic rendering technologies, a surging research interest has been observed in the fields of 3D scene acquisition, representation and re-rendering, make live 3D streaming, real-time 3D conferencing, holographic media, augmented reality, and the metaverse become emerging applications in the near future.

%With the development of photo-realistic rendering technologies, the re-rendering of dynamic 3D scene or novel view can be achieved by densely and simultaneously capturing multi-view image sequences by exposure-synchronized multi-camera system.  At present, the prevailing real-time 3D information acquisition systems, e.g., camera array, light field camera, etc, for realistic scenes necessitate extensive camera resources on the hardware side to gather multi-angle, unobstructed, and redundant-angled colored or depth data. Subsequently, in the software domain, explicit or implicit 3D models are constructed, leveraging high-performance computing devices for computationally intensive operations. These operations involve optimizing unknown parameters for the 3D models, with the ultimate goal of storing, transmitting, and rendering details such as structure, texture, lighting, and radiometry.

%Over an extended period, research efforts in enhancing the overall performance of 3D acquisition and representation systems have adhered to a “more is better” mindset, where, on the hardware front, increased camera resources are thought to facilitate denser viewpoint sampling, and in the software realm, larger parameter sets are employed for more intricate model expressions. Concomitantly, the system’s soft and hardware configuration and deployment may exhibit “less restraints” in certain circumstances. Firstly, redundant resource allocation exists, indicating that judiciously reducing resources does not significantly degrade the system’s performance. Secondly, it is important to note that simply scaling up the hardware and software resources does not lead to unlimited performance gains, as there is a plateau. Thirdly, optimizing critical parameters has a pronounced impact on enhancing system performance. We observe that the 3D acquisition and expression systems exhibit a characteristic diminishing marginal returns: as more resources are invested, the incremental performance gains gradually decline, and targeted optimization of key configuration parameters can trigger substantial leaps in overall performance.

%In academic writing:



%However, in current scientific theories, there is a lack of solid theoretical ground and optimization techniques specifically addressing the mathematical relationship between discrete hardware and software configuration parameters and the performance of 3D information acquisition and expression systems. First, there is an absence of clear methodological guidance for designing loss functions and optimization strategies, which are fundamental components. Second, optimizing general discrete set parameters often confronts non-continuous NP-hard problems, where finding the global optimum involves exhaustive enumeration of all possible parameter combinations, rendering it impractical in real-world scenarios. As a result, the objectives of “efficient resource allocation” and “maximizing effectiveness” during system deployment become challenging, hindering the deployment of applications such as holographic video and virtual reality in flexible and scalable environments. These applications struggle to be deployed with targeted performance upgrades and optimizations, and their performance ceilings remain elusive.

%In academic terms:

%Past research efforts have, predominantly, pursued a “quantity over quality” strategy in enhancing the overall system performance of 3D acquisition and rendering, assuming that augmenting hardware with additional camera resources and software with larger parameter sets leads to greater detail and precision. Meanwhile, there is a possibility, inherent in the system, that resource allocation displays redundancy: although minimizing resources can impact performance minimally. Simultaneously, unyielding linear performance gains are not guaranteed by mere increases in hardware and software capacity, as it appears to operate within an upper bound. It is evident that the systems exhibit a marginal decrement in benefits: the incremental performance gain diminishes with continued investment of resources, and targeted optimization of crucial parameters can bring about significant leaps in system efficiency. This empirical observation underscores the importance of rigorous optimization strategies in these systems to optimize key metrics and counteract diminishing returns.

%Nonetheless, the existing theoretical framework falls short in offering explicit support for optimizing the intricate mathematical relationship between those discrete hardware-software configurational parameters critical to 3D information processing systems. Firstly, the lack of methodological guidance obstructs the development of effective loss functions and optimization strategies, both of which are paramount to the system’s design. Second, common optimizations of discrete parameter sets are inherently non-convex NP-hard problems, necessitating an exhaustive search for the global optimum, rendering such an approach infeasible in practical implementations.

%Consequently, the ambitious goals of resource-efficient “bang for the buck” and targeted performance enhancement using “precise resource allocation” become barriers in the real-world deployment of 3D information-enhanced applications, like holographic projections and virtual reality. These applications face difficulties integrating seamlessly into adaptable and scalable platforms, preventing systematic and targeted improvements in their performance. As such, the fundamental limits of these systems remain largely uncharted territory.

%As such, addressing the mathematical modeling and combinatorial optimization of discrete set parameters in 3D information acquisition and expression systems with a systems approach holds significant theoretical and practical implications. To begin with, a mathematical model of the general relationship between discrete parameters and system performance offers theoretical guidance for quantitative design and optimization of such systems. This allows for systematic efficiency improvements in both software and hardware, and enables the calculation of optimal system configurations under limited resources, thereby delimiting the system’s potential performance limits.

%Furthermore, in the context of diminishing marginal returns, the inherent submodularity of the system can be exploited for efficient optimization. By employing algorithms with optimal system performance growth in polynomial time complexity, one can navigate the intractable complexities of NP-hard problems typically inherent to such optimizations. These methods provide excellent approximation ratios, facilitating the joint optimization of multiple discrete parameters.

%Lastly, the discrete nature of system parameter combinations inherently stands out from continuous system optimization methodologies. Discrete optimization problems align more closely with the practical challenges encountered in daily life and industry, especially inlarge-scale systems where complex parameters lack clear mathematical correlations. By directly optimizing the discrete parameters, it is possible to compute both theoretical limits and optimal solutions without sacrificing the accuracy of problem representation. This approach yields superior design solutions and optimal system configurations, proving invaluable for the meticulous optimization of large-scale projects, underscoring the pivotal role of discrete parameter optimization.