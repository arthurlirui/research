
	
\section*{Entropy Rate of Spherical Harmonics on a Surface}

\subsection*{1. Understanding Spherical Harmonics}

Spherical harmonics \( Y_l^m(\theta, \phi) \) are a set of orthogonal functions defined on the surface of a sphere. They are used in various fields, including physics and engineering, to solve problems with spherical symmetry. The general form is:
\[
Y_l^m(\theta, \phi) = \sqrt{\frac{(2l+1)}{4\pi} \frac{(l-m)!}{(l+m)!}} P_l^m(\cos \theta) e^{im\phi}
\]
where \( P_l^m \) are the associated Legendre polynomials, \( \theta \) is the polar angle, and \( \phi \) is the azimuthal angle.

\subsection*{2. Entropy Rate}

Entropy rate is a measure of the uncertainty per unit time in a stochastic process. For a continuous stochastic process \( X(t) \), the entropy rate \( H(X) \) can be defined as:
\[
H(X) = \lim_{T \to \infty} \frac{1}{T} H(X_{[0,T]})
\]
where \( H(X_{[0,T]}) \) is the differential entropy of the process over the interval \([0, T]\).

\subsection*{3. Stochastic Process on a Sphere}

To apply this concept to spherical harmonics on a surface, we need to consider a stochastic process defined on the spherical surface. Suppose \( Z(\theta, \phi) \) is a random field on the sphere that can be expanded in terms of spherical harmonics:
\[
Z(\theta, \phi) = \sum_{l=0}^{\infty} \sum_{m=-l}^{l} a_l^m Y_l^m(\theta, \phi)
\]
where \( a_l^m \) are random coefficients.

\subsection*{4. Calculating the Entropy Rate}

The entropy rate of \( Z(\theta, \phi) \) can be approached by analyzing the entropy rate of the coefficients \( a_l^m \). If we assume the coefficients \( a_l^m \) are independent and identically distributed (i.i.d.) random variables with some known distribution, the entropy rate can be computed from the distribution of these coefficients.

\paragraph*{1. Distribution of \( a_l^m \):}
If \( a_l^m \) follows a normal distribution \( \mathcal{N}(0, \sigma^2) \), the entropy \( H(a_l^m) \) for each coefficient is given by:
\[
H(a_l^m) = \frac{1}{2} \log (2\pi e \sigma^2)
\]

\paragraph*{2. Summing the Entropies:}
Since there are \((2l+1)\) coefficients for each \( l \), the total entropy for a given \( l \) is:
\[
H_l = (2l + 1) \cdot \frac{1}{2} \log (2\pi e \sigma^2)
\]

\paragraph*{3. Entropy Rate:}
If we consider the total entropy up to a certain \( l_{\text{max}} \), the entropy rate \( H(Z) \) can be approximated by:
\[
H(Z) = \lim_{l_{\text{max}} \to \infty} \frac{1}{4\pi} \sum_{l=0}^{l_{\text{max}}} (2l + 1) \cdot \frac{1}{2} \log (2\pi e \sigma^2)
\]
Simplifying, we get:
\[
H(Z) = \frac{1}{4\pi} \sum_{l=0}^{\infty} (2l + 1) \cdot \frac{1}{2} \log (2\pi e \sigma^2)
\]


	
\section*{Spherical Harmonic Functions}

Spherical harmonic functions are a set of orthogonal functions defined on the surface of a sphere. They play a crucial role in various areas of physics, mathematics, and engineering, particularly in solving problems with spherical symmetry, such as in quantum mechanics, geophysics, and computer graphics.

\subsection*{Definition}

A spherical harmonic function \( Y_l^m(\theta, \phi) \) is defined by two integers:
\begin{itemize}
	\item \( l \): the degree, which is a non-negative integer ( \( l = 0, 1, 2, \ldots \) ).
	\item \( m \): the order, which is an integer such that \( -l \leq m \leq l \).
\end{itemize}

The general form of a spherical harmonic is given by:
\[
Y_l^m(\theta, \phi) = \sqrt{\frac{(2l+1)}{4\pi} \frac{(l-m)!}{(l+m)!}} P_l^m(\cos \theta) e^{im\phi}
\]

where:
\begin{itemize}
	\item \( \theta \) is the polar angle (colatitude) with \( 0 \leq \theta \leq \pi \).
	\item \( \phi \) is the azimuthal angle (longitude) with \( 0 \leq \phi \leq 2\pi \).
	\item \( P_l^m \) are the associated Legendre polynomials.
\end{itemize}

\subsection*{Associated Legendre Polynomials}

The associated Legendre polynomials \( P_l^m(x) \) are defined for \( -1 \leq x \leq 1 \) and are related to the Legendre polynomials \( P_l(x) \) by:
\[
P_l^m(x) = (1 - x^2)^{\frac{m}{2}} \frac{d^m}{dx^m} P_l(x)
\]

The Legendre polynomials \( P_l(x) \) themselves are solutions to the Legendre differential equation:
\[
\frac{d}{dx} \left( (1 - x^2) \frac{d P_l(x)}{dx} \right) + l(l+1) P_l(x) = 0
\]

\subsection*{Orthogonality}

Spherical harmonics are orthogonal functions on the surface of a sphere. The orthogonality condition is expressed as:
\[
\int_0^{2\pi} \int_0^{\pi} Y_l^m(\theta, \phi) Y_{l'}^{m'}(\theta, \phi)^* \sin \theta \, d\theta \, d\phi = \delta_{ll'} \delta_{mm'}
\]
where \( \delta_{ll'} \) and \( \delta_{mm'} \) are the Kronecker delta functions, and \( Y_{l'}^{m'}(\theta, \phi)^* \) is the complex conjugate of \( Y_{l'}^{m'}(\theta, \phi) \).

\subsection*{Properties}

\begin{enumerate}
	\item \textbf{Normalization:}
	Spherical harmonics are normalized such that:
	\[
	\int_0^{2\pi} \int_0^{\pi} |Y_l^m(\theta, \phi)|^2 \sin \theta \, d\theta \, d\phi = 1
	\]
	
	\item \textbf{Completeness:}
	Any square-integrable function on the sphere can be expanded in terms of spherical harmonics. This is known as spherical harmonic expansion:
	\[
	f(\theta, \phi) = \sum_{l=0}^{\infty} \sum_{m=-l}^{l} a_l^m Y_l^m(\theta, \phi)
	\]
	where \( a_l^m \) are the spherical harmonic coefficients given by:
	\[
	a_l^m = \int_0^{2\pi} \int_0^{\pi} f(\theta, \phi) Y_l^m(\theta, \phi)^* \sin \theta \, d\theta \, d\phi
	\]
	
	\item \textbf{Symmetry:}
	Spherical harmonics exhibit symmetry properties such as:
	\begin{itemize}
		\item Conjugate symmetry: \( Y_l^{-m}(\theta, \phi) = (-1)^m Y_l^m(\theta, \phi)^* \)
		\item Parity: \( Y_l^m(\pi - \theta, \phi + \pi) = (-1)^l Y_l^m(\theta, \phi) \)
	\end{itemize}
\end{enumerate}

\subsection*{Applications}

\begin{itemize}
	\item \textbf{Quantum Mechanics:} Used in solving the Schrödinger equation for atoms and molecules, particularly for the angular part of wavefunctions.
	\item \textbf{Geophysics:} Applied in representing the Earth's gravitational and magnetic fields.
	\item \textbf{Computer Graphics:} Utilized in lighting calculations and in the representation of spherical data.
\end{itemize}

In summary, spherical harmonics are fundamental in the analysis and representation of functions defined on the sphere, providing a powerful tool in both theoretical and applied sciences.


