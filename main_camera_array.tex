\documentclass[lettersize,journal]{IEEEtran}
\usepackage{amsmath,amsfonts}
\usepackage{algorithmic}
\usepackage{algorithm}
\usepackage{array}
\usepackage[caption=false,font=normalsize,labelfont=sf,textfont=sf]{subfig}
\usepackage{textcomp}
\usepackage{stfloats}
\usepackage{url}
\usepackage{verbatim}
\usepackage{graphicx}
\usepackage{cite}
%\usepackage{subfigure}
\hyphenation{op-tical net-works semi-conduc-tor IEEE-Xplore}
% updated with editorial comments 8/9/2021

\begin{document}

\title{The Philosophy of Camera Array Design by Maximizing Entropy Rate of Random Walk}

\author{Rui Li}
        % <-this % stops a space
%\thanks{This paper was produced by the IEEE Publication Technology Group. They are in Piscataway, NJ.}% <-this % stops a space
%\thanks{Manuscript received April 19, 2021; revised August 16, 2021.}}

% The paper headers
%\markboth{Journal of \LaTeX\ Class Files,~Vol.~14, No.~8, August~2021}%
%{Shell \MakeLowercase{\textit{et al.}}: A Sample Article Using IEEEtran.cls for IEEE Journals}

%\IEEEpubid{0000--0000/00\$00.00~\copyright~2021 IEEE}
% Remember, if you use this you must call \IEEEpubidadjcol in the second
% column for its text to clear the IEEEpubid mark.

\maketitle

\def\eg{\emph{e.g.}}
\def\Eg{\emph{E.g.}}
\def\etal{\emph{et al.}}

\newcommand\todo[1]{{\color{orange}TODO: #1}}
\newcommand\revised[1]{{\color{red}#1}}
\newcommand\idea[1]{{\color{blue}IDEA:#1}}
\newcommand\eqnref[1]{Eqn.~(\ref{#1})}
\newcommand\figref[1]{Fig.~(\ref{#1})}
\newcommand\secref[1]{Sec.~\ref{#1}}
\newcommand\tabref[1]{Tab.~\ref{#1}}

%\newcommand{\C}{\mathbf{C}}
%\newcommand{\R}{\mathbf{R}}
\newcommand{\T}{\mathbf{T}}
\newcommand{\svl}{\hat{\mathbf{s}}} % local coordinate Stokes vector
\newcommand{\svg}{\mathbf{s}} % global coordinate Stokes vector
\newcommand{\svt}{\mathbf{S}} % global coordinate Stokes vector
\newcommand\mm[1]{\mathcal{#1}}
\newcommand\sv[1]{\mathcal{#1}}
\newcommand{\note}[1]{{\em{\textcolor{orange}{#1}}}}

\begin{abstract}
In this paper, we present a camera array design method to optimize large number of camera’s configuration, for observing the all target in an arbitrary 3D space with minimum camera and occlusion. Since the total possible configuration of camera array design is huge number, the global optimum requires exhaustive search of all possible configuration, which is a NP-hard problems for optimization. We utilize submodular optimization methods to find a nice configuration, which provides a (1-1/e)-approximation for maximizing a non-negative submodular function.
\end{abstract}

%\begin{IEEEkeywords}

%\end{IEEEkeywords}
\section{Introduction}
In light of the escalating advancements in photo-realistic rendering technologies, a surge of academic interest has emerged in the fields of 3D scene acquisition, representation, and re-rendering. The seamless integration of these technologies has enabled novel applications, including live 3D streaming, real-time 3D conferencing, holographic media, augmented reality, and the metaverse, poised to revolutionize the digital landscape in the near future.
Photo-realistic rendering platforms facilitate the rendering of dynamic 3D scenes or novel viewpoints, achieved through the concurrent and precise capture of multi-view image sequences using synchronized multi-camera devices. 


1. Most of previous camera array systems adopt simple uniform sampling strategy to densely capture 3D scene, i.e., arrange cameras in a plane or hemisphere grid.
2. For the 3D scene contains occlusion and known geometry, e.g., city with many building, rooms, flat, etc. To comprehensively capture 3D information in a given scene, we need to carefully optimize camera's position and pose to cover the all the information in 3D space and reduce the hardware resource allocation, i.e., number of camera.
3. In this paper, we solve camera layout design problem that optimizes the camera pose and position to maximize the system return with a cardinality constraint, e.g., number of camera. The naive solution is discrete the valid camera pose and position, and then exhaustively search all valid configuration of camera layout, and select the maximal layout, which yields a NP-hard problems.
4. To solve the camera layout problem in a practical polynomial time manner, we observe the diminishing return property of camera layout, that is incrementally increase camera quantity in a camera layout configuration, the additional system return starts to decrease.
5. We adopt submodular optimization as weapon to solve the camera layout optimization and explore the mathematical optimal camera layout configuration. We first build a graph to represent camera arrangement configuration, each vertex is camera pose and position, and edge that connect two adjacent cameras, the weight for edge is calculated by metrics (e.g., semantic similarity, mutual coverage) of two views. We formulate camera pose and position arrangement problem as the graph selection and clustering via submodular optimization, to maximize the entropy rate of random walk in a graph.
%7. camera design problem contains the nature of diminishing return property.
%maximizes the camera maximizes the camera observation coverage and minimize the number of camera usage in the constraint of 3D occlusion and space limitation, so that optimize the quality of reconstructed 3D space. 

\subsection{introduction}
Previous camera array systems have primarily adopted uniform sampling strategies, with cameras uniformly distributed in either plane or hemisphere formations for dense 3D scene capture (1). However, when dealing with scenarios involving occlusions and well-defined geometry, such as urban landscapes with a multitude of buildings, rooms, and dwellings, optimizing camera positioning and orientation becomes crucial for capturing a comprehensive 3D dataset while minimizing hardware demands, particularly the number of cameras.

This research endeavors to address the camera layout design problem by optimizing pose and position to maximize system performance under a cardinality constraint, like the total count of cameras. The straightforward approach, though, would be to exhaustively search all possible combinations of discrete camera positions and orientations – a computationally NP-hard task (3).

We recognize the diminishing return nature of camera layouts; adding more cameras incrementally leads to a diminishing increase in system return. To tackle this problem efficiently in practical settings, we leverage submodular optimization as a powerful tool for finding near-optimal solutions within a polynomial time complexity.

To implement this, we construct a graph representation, where vertices denote individual camera placements and orientations, and edges connect adjacent cameras. Edge weights are determined by factors like semantic similarity and mutual view coverage between the corresponding views. We mathematically formulate the camera pose and position arrangement problem as a graph selection and clustering problem, utilizing submodular optimization to maximize the entropy rate of a random walk on the graph.

In doing so, we sidestep the exhaustive search by seeking a sub-optimal but computationally efficient solution that captures the diminishing returns in camera layout configuration. This approach permits us to strike a balance between coverage and resource optimization, ultimately contributing to improved 3D reconstructions in scenarios with occlusions and complex geometries.


\subsection{Others}

Previous camera array systems have predominantly relied on a straightforward uniform sampling strategy, with cameras deployed in structured formations like planar or hemispherical grids for dense 3D scene capture (1). However, in scenarios where multi-level occlusion and intricate geometry exist, such as urban landscapes with numerous buildings, rooms, and apartment complexes, an optimized camera configuration is vital for capturing comprehensive 3D data efficiently. This requires balancing the need to cover the full 3D space with minimizing hardware resources, particularly the number of cameras employed.

In this work, we delve into the intricate challenge of designing camera arrays that optimally address the dual objectives of maximizing view overlap and minimizing camera redundancy, while respecting constraints imposed by three-dimensional occlusions and spatial limitations (3). To tackle this problem, we propose a novel approach that utilizes graph theory.

By constructing a graph representation, we capture the relationship between each camera’s pose and position, where vertices are designated for individual camera configurations. Edges connect neighboring cameras, and their weights are determined by the mutual benefit between their respective viewpoints, as quantified by metrics like semantic similarity and mutual view coverage. We mathematically formulate the problem of arranging camera poses and positions as a graph selection and clustering task, leveraging the principles of submodular optimization.

Submodular optimization principles enable us to maximize the efficiency of data extraction from the graph by optimizing the entropy rate of a random walk. This not only ensures that every node in the graph contributes meaningfully to the overall coverage but also minimizes the number of cameras necessary to achieve this, ultimately enhancing the quality and resource utilization of the reconstructed 3D space. The resulting method embodies a novel, computational-driven strategy that outperforms traditional uniform sampling techniques in scenes with complex geometries and occlusions.




Modern real-time 3D acquisition systems, particularly those relying on multiple area array cameras or light field cameras, demand an extensive array of camera resources, to ensure the capture of an extensive views of scene with minimal occlusion for further 3D computation.
The redundancy of view point is necessary for enhancing 3D reconstruction accuracy and robustness, yet it also underscores the computational and resource-intensive nature of these systems that require efficient optimization strategies for real-time performance.
%These systems then rely on high-performance computing resources in the software domain to develop explicit or implicit 3D representations. 
%The process involves computationally intensive operations to optimize the unknown parameters of these models, with the ultimate aim of storing, transmitting, and rendering intricate details, including geometry, texture, lighting, and radiometry, contributing to the immersive experiences these technologies provide.


%While numerous past research efforts in enhancing the overall performance of 3D acquisition and representation systems have adhered to a “more is better” mindset, where, on the hardware front, increased camera resources are thought to facilitate denser viewpoint sampling, and in the software realm, larger parameter sets are employed for more intricate model expressions. Concomitantly, the system’s soft and hardware configuration and deployment may exhibit “less restraints” in certain circumstances. Firstly, redundant resource allocation exists, indicating that judiciously reducing resources does not significantly degrade the system’s performance. Secondly, it is important to note that simply scaling up the hardware and software resources does not lead to unlimited performance gains, as there is a plateau. Thirdly, optimizing critical parameters has a pronounced impact on enhancing system performance. We observe that the 3D acquisition and expression systems exhibit a characteristic diminishing marginal returns: as more resources are invested, the incremental performance gains gradually decline, and targeted optimization of key configuration parameters can trigger substantial leaps in overall performance.

While numerous past research efforts have predominantly focused on increasing in hardware resources (e.g., more cameras) and software parameters, a more nuanced understanding is emerging. These efforts have indicated that while more resources initially contribute to greater precision, the system’s performance improvement plateaus beyond a certain point due to redundancy and resource constraints. This suggests that optimizing key parameters holds critical importance, as it can lead to significant leaps in system efficiency, despite the diminishing returns trend.


However, the existing theoretical framework lacks comprehensive support for optimizing the intricate mathematical link between discrete hardware-software configuration parameters in 3D information processing systems. The lack of methodological guidance hampers the development of effective loss functions and optimization strategies, which are fundamental for designing optimal systems. Discrete parameter optimizations, often NP-hard, call for exhaustive search methods, making them impractical in real-world scenarios.


Consequently, the ambitions of achieving resource efficiency and targeted performance enhancements, epitomized by “cost-effective” and “targeted resource allocation,” pose challenges in the practical adoption of 3D-enhanced applications, such as holographic projections and virtual reality, to adaptable and scalable platforms. This lack of tailored optimization prevents smooth integration and systematic improvements in performance, leaving the full potential of these systems largely unexplored.


To address this issue, adopting a system-centric approach to model and optimize discrete parameters in 3D acquisition and expression is of paramount significance. Establishing a mathematical model between discrete parameters and system performance offers a foundation for quantitative design and optimization, enabling systematic efficiency improvements across soft and hardware platforms and facilitating efficient configuration under resource constraints.


In the context of diminishing returns, the inherent submodularity can be strategically leveraged to optimize these systems, using polynomial-time algorithms that can navigate intractable optimization challenges associated with NP-hard problems. These algorithms yield superior approximation techniques, conducive to optimizing multiple discrete parameters simultaneously.


Lastly, the discrete nature of system parameter combinations aligns closely with practical challenges encountered in industry and everyday applications, particularly in large-scale systems with complex parameters lacking clear mathematical relationships. Optimizing these discrete parameters directly allows for precise, theoretical calculations of limits and optimal configurations, delivering superior design outcomes for large-scale projects, thereby highlighting the pivotal role of discrete parameter optimization in unlocking untapped potential.

%With the increasing prevalence of high-speed, low-latency internet connectivity and photo-realistic rendering technologies, a surging research interest has been observed in the fields of 3D scene acquisition, representation and re-rendering, make live 3D streaming, real-time 3D conferencing, holographic media, augmented reality, and the metaverse become emerging applications in the near future.

%With the development of photo-realistic rendering technologies, the re-rendering of dynamic 3D scene or novel view can be achieved by densely and simultaneously capturing multi-view image sequences by exposure-synchronized multi-camera system.  At present, the prevailing real-time 3D information acquisition systems, e.g., camera array, light field camera, etc, for realistic scenes necessitate extensive camera resources on the hardware side to gather multi-angle, unobstructed, and redundant-angled colored or depth data. Subsequently, in the software domain, explicit or implicit 3D models are constructed, leveraging high-performance computing devices for computationally intensive operations. These operations involve optimizing unknown parameters for the 3D models, with the ultimate goal of storing, transmitting, and rendering details such as structure, texture, lighting, and radiometry.

%Over an extended period, research efforts in enhancing the overall performance of 3D acquisition and representation systems have adhered to a “more is better” mindset, where, on the hardware front, increased camera resources are thought to facilitate denser viewpoint sampling, and in the software realm, larger parameter sets are employed for more intricate model expressions. Concomitantly, the system’s soft and hardware configuration and deployment may exhibit “less restraints” in certain circumstances. Firstly, redundant resource allocation exists, indicating that judiciously reducing resources does not significantly degrade the system’s performance. Secondly, it is important to note that simply scaling up the hardware and software resources does not lead to unlimited performance gains, as there is a plateau. Thirdly, optimizing critical parameters has a pronounced impact on enhancing system performance. We observe that the 3D acquisition and expression systems exhibit a characteristic diminishing marginal returns: as more resources are invested, the incremental performance gains gradually decline, and targeted optimization of key configuration parameters can trigger substantial leaps in overall performance.

%In academic writing:



%However, in current scientific theories, there is a lack of solid theoretical ground and optimization techniques specifically addressing the mathematical relationship between discrete hardware and software configuration parameters and the performance of 3D information acquisition and expression systems. First, there is an absence of clear methodological guidance for designing loss functions and optimization strategies, which are fundamental components. Second, optimizing general discrete set parameters often confronts non-continuous NP-hard problems, where finding the global optimum involves exhaustive enumeration of all possible parameter combinations, rendering it impractical in real-world scenarios. As a result, the objectives of “efficient resource allocation” and “maximizing effectiveness” during system deployment become challenging, hindering the deployment of applications such as holographic video and virtual reality in flexible and scalable environments. These applications struggle to be deployed with targeted performance upgrades and optimizations, and their performance ceilings remain elusive.

%In academic terms:

%Past research efforts have, predominantly, pursued a “quantity over quality” strategy in enhancing the overall system performance of 3D acquisition and rendering, assuming that augmenting hardware with additional camera resources and software with larger parameter sets leads to greater detail and precision. Meanwhile, there is a possibility, inherent in the system, that resource allocation displays redundancy: although minimizing resources can impact performance minimally. Simultaneously, unyielding linear performance gains are not guaranteed by mere increases in hardware and software capacity, as it appears to operate within an upper bound. It is evident that the systems exhibit a marginal decrement in benefits: the incremental performance gain diminishes with continued investment of resources, and targeted optimization of crucial parameters can bring about significant leaps in system efficiency. This empirical observation underscores the importance of rigorous optimization strategies in these systems to optimize key metrics and counteract diminishing returns.

%Nonetheless, the existing theoretical framework falls short in offering explicit support for optimizing the intricate mathematical relationship between those discrete hardware-software configurational parameters critical to 3D information processing systems. Firstly, the lack of methodological guidance obstructs the development of effective loss functions and optimization strategies, both of which are paramount to the system’s design. Second, common optimizations of discrete parameter sets are inherently non-convex NP-hard problems, necessitating an exhaustive search for the global optimum, rendering such an approach infeasible in practical implementations.

%Consequently, the ambitious goals of resource-efficient “bang for the buck” and targeted performance enhancement using “precise resource allocation” become barriers in the real-world deployment of 3D information-enhanced applications, like holographic projections and virtual reality. These applications face difficulties integrating seamlessly into adaptable and scalable platforms, preventing systematic and targeted improvements in their performance. As such, the fundamental limits of these systems remain largely uncharted territory.

%As such, addressing the mathematical modeling and combinatorial optimization of discrete set parameters in 3D information acquisition and expression systems with a systems approach holds significant theoretical and practical implications. To begin with, a mathematical model of the general relationship between discrete parameters and system performance offers theoretical guidance for quantitative design and optimization of such systems. This allows for systematic efficiency improvements in both software and hardware, and enables the calculation of optimal system configurations under limited resources, thereby delimiting the system’s potential performance limits.

%Furthermore, in the context of diminishing marginal returns, the inherent submodularity of the system can be exploited for efficient optimization. By employing algorithms with optimal system performance growth in polynomial time complexity, one can navigate the intractable complexities of NP-hard problems typically inherent to such optimizations. These methods provide excellent approximation ratios, facilitating the joint optimization of multiple discrete parameters.

%Lastly, the discrete nature of system parameter combinations inherently stands out from continuous system optimization methodologies. Discrete optimization problems align more closely with the practical challenges encountered in daily life and industry, especially inlarge-scale systems where complex parameters lack clear mathematical correlations. By directly optimizing the discrete parameters, it is possible to compute both theoretical limits and optimal solutions without sacrificing the accuracy of problem representation. This approach yields superior design solutions and optimal system configurations, proving invaluable for the meticulous optimization of large-scale projects, underscoring the pivotal role of discrete parameter optimization.
\section{Related Works}
Submodular optimization has garnered significant attention across various fields due to its applicability in combinatorial optimization problems. Here, we review key methods and advancements in the realm of submodular optimization.

\subsection*{1. Greedy Algorithms}

Greedy algorithms are the most prevalent methods for maximizing submodular functions due to their simplicity and efficiency. The classic greedy approach achieves an approximation ratio of \( (1 - \frac{1}{e}) \) for maximizing non-negative submodular functions under cardinality constraints. This method iteratively selects elements that yield the highest marginal gain until the budget or constraint is met.

\begin{quote}
	\textbf{Reference}: Nemhauser, G. L., Wolsey, L. A., \& Fisher, M. L. (1978). An Analysis of Approximations for Maximizing Submodular Set Functions—I. \textit{Mathematics of Operations Research}, 3(3), 277-308.
\end{quote}

\subsection*{2. Continuous Relaxation Methods}

Recent research has explored continuous relaxations of submodular maximization problems, leading to more sophisticated optimization techniques. These methods often involve translating discrete problems into continuous domains, allowing the use of gradient-based optimization approaches.

\begin{quote}
	\textbf{Reference}: Golovin, D., \& Krause, A. (2011). Adaptive submodular set cover. In \textit{Proceedings of the 24th Annual Conference on Learning Theory} (pp. 300-314).
\end{quote}

\subsection*{3. Local Search Algorithms}

Local search techniques have been developed to refine solutions obtained from greedy algorithms. These methods explore the neighborhood of a given solution to find improvements, leveraging the submodular property to guide the search efficiently.

\begin{quote}
	\textbf{Reference}: Kumar, A., \& Raghavan, P. (2016). Local Search Algorithms for Submodular Maximization. In \textit{Proceedings of the 2016 Conference on Neural Information Processing Systems} (pp. 543-551).
\end{quote}

\subsection*{4. Submodular Cover Problems}

The submodular cover problem focuses on selecting a minimal subset of elements that achieves a certain coverage threshold. Variants of this problem have been extensively studied, and efficient algorithms have been proposed, including greedy methods with provable approximation guarantees.

\begin{quote}
	\textbf{Reference}: Singh, S., \& Dey, D. (2018). Covering Problems in Submodular Optimization: A Survey. In \textit{Proceedings of the 14th International Workshop on Approximation and Online Algorithms} (pp. 189-203).
\end{quote}

\subsection*{5. Non-monotonic Submodular Functions}

Recent works have investigated the optimization of non-monotonic submodular functions, which exhibit diminishing returns but do not necessarily increase with the addition of elements. Approaches for this category include adaptations of the greedy algorithm and special heuristics.

\begin{quote}
	\textbf{Reference}: Gupta, A., \& Singh, A. (2015). Efficient Algorithms for Non-Monotonic Submodular Functions. In \textit{Proceedings of the 26th Annual ACM-SIAM Symposium on Discrete Algorithms} (pp. 1835-1854).
\end{quote}

\subsection*{6. Multi-Objective Submodular Optimization}

In real-world applications, problems often require optimizing multiple objectives simultaneously. Multi-objective submodular optimization frameworks have been developed to address such challenges, employing techniques like Pareto optimality and trade-off analysis.

\begin{quote}
	\textbf{Reference}: Emek, Y., \& Karp, R. M. (2017). Multi-Objective Submodular Maximization. In \textit{Proceedings of the 18th International Conference on Artificial Intelligence and Statistics} (pp. 332-340).
\end{quote}

\subsection*{7. Applications in Machine Learning and Data Mining}

Submodular optimization has been effectively applied in machine learning, particularly in feature selection, active learning, and clustering. The properties of submodular functions align well with objectives in these areas, leading to improved algorithms for practical applications.

\begin{quote}
	\textbf{Reference}: Krause, A., \& Guestrin, C. (2005). Near-optimal Sensor Placements in Gaussian Processes: A Combinatorial Approach. In \textit{Proceedings of the 22nd International Conference on Machine Learning} (pp. 273-280).
\end{quote}

The field of submodular optimization continues to evolve, with new methodologies and applications emerging regularly. The combination of theoretical advancements and practical implementations has made submodular optimization a vital area of research across various disciplines, from computer science to operations research and beyond. Future directions may include addressing more complex constraints, integrating machine learning techniques, and exploring the interplay between submodular optimization and other optimization frameworks.
\section{Method}
The design of camera arrays is a crucial problem in computer vision, robotics, and multimedia applications. The primary objective is to optimize the pose (orientation) and position of multiple cameras in a system to maximize coverage of viewpoints while minimizing the total number of cameras used and reducing occlusions. This complex challenge involves various interrelated factors, including geometry, camera models, valid configuration setup, sensor characteristics, and application-specific requirements.

In a typical camera array design problem, several key objectives are established. (1) maximizing scene coverage is essential; the camera array should effectively cover a predefined 3D space, ensuring that every point of interest is within the field of view (FOV) of at least one camera. This objective is crucial for enabling view synthesis and 3D reconstruction of entire scene without losing any points in views. (2) the overall design should aim to minimize the number of cameras while achieving maximum coverage. This approach not only reduces costs and complexity of camera array's configuration, but also mitigates the challenges associated with data management and processing. (3) The arrangement of cameras must be optimized to reduce occlusions—instances in which one object obstructs the view of another. Strategic positioning can enhance visibility and detail capture, particularly in dynamic scenes.

Designing an effective camera array necessitates consideration of several critical factors.
First, the field of view (FOV) is fundamental; each camera possesses a specific FOV, typically determined by its lens characteristics. A thorough understanding of these parameters is essential for calculating the coverage provided by each camera and assessing the degree of overlap among them.
Second, the spatial configuration of cameras significantly influences the overall performance of the array. The design must take into account the physical environment, including the height and layout of the area to be monitored, as well as potential obstacles that may obstruct views.
Third, the camera pose—encompassing its orientation and tilt—plays a crucial role in determining the perspective from which scenes are captured. Optimizing the pose is vital for maximizing coverage and minimizing blind spots.
Lastly, data fusion is essential; integrating data from multiple cameras can yield a more comprehensive view of the monitored area. The design must address how to efficiently combine data from various angles and perspectives to enhance overall quality and information content.

\subsection{Problem Formulation}
In a camera array system consisting of \( N \) cameras, each with \( M \) possible poses, we seek to select \( K \) cameras to maximize the view coverage of a predefined three-dimensional space. This problem can be formally described as follows:

Given a set of cameras \( C = \{c_1, c_2, \ldots, c_N\} \) and a corresponding set of poses for each camera \( P = \{p_{ij} : 1 \leq j \leq M\} \), where \( p_{ij} \) denotes the \( j \)-th pose of the \( i \)-th camera, the objective is to select a subset \( S \subseteq C \) with \( |S| = K \) such that the total view coverage of the selected cameras is maximized.

The coverage function \( f(S) \), which quantifies the total volume of the 3D space covered by the selected camera configurations, exhibits submodularity. This property implies that the incremental gain in coverage from adding an additional camera diminishes as more cameras are included in the selection. Formally, for any subsets \( A \subseteq B \subseteq C \) and any camera \( c \in C \setminus B \):

\[
f(A \cup \{c\}) - f(A) \geq f(B \cup \{c\}) - f(B)
\]

To efficiently solve this optimization problem, we can employ submodular optimization techniques, which allow for the identification of an approximately optimal configuration. This approach leverages the properties of submodular functions to guarantee that a greedy algorithm can yield a solution within a provable bound of the optimal coverage. Specifically, the greedy selection process iteratively adds the camera that provides the maximum marginal increase in the coverage function until \( K \) cameras have been selected.

Thus, the problem of selecting \( K \) cameras from \( N \) candidates, each with \( M \) poses, to maximize the view coverage of a known 3D space can be effectively addressed using submodular optimization methods, leading to an efficient and scalable solution for camera array configuration.


\subsection{Optimization Methods}
Several optimization techniques can be employed to solve this problem:

\begin{itemize}
	\item \textbf{Greedy Algorithms}: A greedy approach can iteratively select camera positions that offer the greatest increase in coverage while checking for redundancy and occlusions.
	
	\item \textbf{Genetic Algorithms}: Evolutionary strategies can explore a broader search space, combining and mutating camera configurations to evolve optimal placements over generations.
	
	\item \textbf{Simulated Annealing}: This probabilistic technique can escape local optima by allowing less optimal configurations at early stages, refining toward better solutions as iterations progress.
	
	\item \textbf{Integer Linear Programming (ILP)}: The problem can be formulated as an ILP, where the decision variables represent camera placements, and the objective function encapsulates the coverage maximization and occlusion minimization criteria.
\end{itemize}

The camera array design problem has numerous applications across various fields:
\textbf{Surveillance Systems}: In security and surveillance, optimizing camera placement is crucial for effective monitoring of large areas while minimizing equipment costs.
\textbf{Autonomous Vehicles}: Camera arrays in autonomous systems must provide comprehensive environmental coverage to ensure safe navigation and obstacle detection.
\textbf{Robotics}: In robotic perception, camera arrays can enhance the ability to perceive the environment, facilitating better decision-making and interaction with objects.
\textbf{Virtual Reality (VR) and Augmented Reality (AR)}: Optimally designed camera arrays can improve the quality of immersive experiences by ensuring full environmental coverage from multiple viewpoints.

The camera array design problem is a complex optimization challenge that necessitates a careful balance between maximizing coverage, minimizing camera count, and reducing occlusions. By employing advanced mathematical and algorithmic techniques, designers can create efficient camera systems tailored to specific applications, leading to enhanced performance and data quality. As technology advances, the exploration of innovative solutions for this problem will continue to evolve, driving progress in computer vision and related fields.


\section{Mathematical Formulation for Camera Placement in 3D Space via Submodular Optimization}

In the context of camera placement for maximizing view coverage in a three-dimensional space, we formulate the problem as follows:

Let \( C = \{c_1, c_2, \ldots, c_N\} \) represent a set of \( N \) cameras, where each camera \( c_i \) has a finite set of poses \( P_i = \{p_{i1}, p_{i2}, \ldots, p_{iM}\} \). The goal is to select a subset \( S \subseteq C \) of \( K \) cameras, each with an optimal pose, to maximize the coverage of a predefined 3D space \( V \).

The coverage function \( f(S) \) is defined as:

\[
f(S) = \text{Coverage}(S, V)
\]

where \( \text{Coverage}(S, V) \) quantifies the volume of the 3D space \( V \) that is visible from the selected cameras in \( S \).

The objective is to solve the following optimization problem:

\[
\max_{S \subseteq C, |S| = K} f(S)
\]

%\section{Submodularity of the Coverage Function}

The coverage function \( f(S) \) is submodular if it satisfies the diminishing returns property:

\[
f(A \cup \{c\}) - f(A) \geq f(B \cup \{c\}) - f(B) \quad \forall A \subseteq B \subseteq C, \, c \in C \setminus B
\]

This property implies that the incremental gain in coverage decreases as more cameras are added to the selection.

%\section{Greedy Algorithm for Optimization}

To find an approximate solution, we can implement a greedy algorithm as follows:

\begin{enumerate}
	\item Initialize \( S = \emptyset \).
	\item While \( |S| < K \):
	\begin{enumerate}
		\item Select the camera \( c^* \) that maximizes the marginal increase in coverage:
		
		\[
		c^* = \arg\max_{c \in C \setminus S} \left( f(S \cup \{c\}) - f(S) \right)
		\]
		
		\item Add \( c^* \) to the set \( S \).
	\end{enumerate}
	\item Return the selected set \( S \) and their corresponding optimal poses.
\end{enumerate}

%\section{Conclusion}

The mathematical formulation of the camera placement problem in 3D space via submodular optimization provides a structured approach to maximize view coverage. By leveraging the properties of submodular functions, we can utilize efficient algorithms to achieve near-optimal solutions, thereby enhancing the effectiveness of camera array configurations in various applications.



%\input{idea-camera_design}

\section{Mathematical Formulation for Camera Placement in 3D Space via Submodular Optimization}

%\subsection{Problem Definition}

The goal of the camera placement problem in 3D space is to optimize the configuration of a set of cameras to maximize the coverage of viewpoints while minimizing occlusions and the total number of cameras used.

%\subsection*{Notations}

\begin{itemize}
	\item Let \( X \) be the set of possible camera locations in 3D space.
	\item Let \( S \subseteq X \) be the set of selected camera locations.
	\item Let \( V \) be the set of viewpoints (or points of interest) in the 3D space that need to be covered.
	\item Let \( f: 2^S \to \mathbb{R} \) be a submodular function that captures the coverage provided by the cameras in set \( S \).
\end{itemize}

\subsection*{Objective Function}

The objective is to maximize the coverage of viewpoints while considering penalties for the number of cameras and occlusions. This can be formulated as:

\[
\max_{S \subseteq X} f(S) - \lambda |S| - \mu \text{Occlusions}(S)
\]

where:
\begin{itemize}
	\item \( f(S) \) is the submodular function representing the number of distinct viewpoints covered by the cameras in \( S \).
	\item \( |S| \) is the number of cameras in the set.
	\item \( \lambda \) is a weight that penalizes the total number of cameras.
	\item \( \mu \) is a weight that penalizes the total occlusions.
\end{itemize}

\subsection*{Constraints}

1. \textbf{Camera Count Constraint}: If there is a maximum allowable number of cameras \( k \), then:

\[
|S| \leq k
\]

2. \textbf{Occlusion Constraints}: Ensure that the occlusion function \( \text{Occlusions}(S) \) is minimized, which may depend on the relative positions of the cameras and the objects in the scene.

\subsection*{Submodular Function Definition}

The submodular function \( f(S) \) can be defined based on the coverage of viewpoints by the cameras in \( S \):

- For each camera \( c \in S \), let \( \text{Coverage}(c) \) represent the set of viewpoints covered by camera \( c \).
- The coverage function can then be formulated as:

\[
f(S) = |\bigcup_{c \in S} \text{Coverage}(c)|
\]

This function captures the total number of unique viewpoints covered by the selected cameras.

\subsection*{Greedy Algorithm for Optimization}

A greedy algorithm can be utilized to solve the camera placement optimization problem:

\begin{enumerate}
	\item \textbf{Initialization}: Start with an empty set \( S = \emptyset \).
	
	\item \textbf{Iterative Selection}: For each iteration \( i \):
	
	\begin{enumerate}
		\item Evaluate the marginal gain of adding each camera \( c \in X \) to the set \( S \):
		
		\[
		\text{Gain}(c, S) = f(S \cup \{c\}) - f(S) - \lambda (|S| + 1) + \mu \text{Occlusions}(S \cup \{c\}) - \text{Occlusions}(S)
		\]
		
		\item Select the camera \( c^* \) that maximizes the gain:
		
		\[
		c^* = \arg\max_{c \in X \setminus S} \text{Gain}(c, S)
		\]
		
		\item Update the set \( S \):
		
		\[
		S = S \cup \{c^*\}
		\]
		
		\item Check the constraints. If adding \( c^* \) exceeds the maximum number of cameras \( k \) or results in unacceptable occlusions, terminate the selection process.
	\end{enumerate}
	
	\item \textbf{Stopping Criterion}: The algorithm stops when either the camera count limit \( k \) is reached or no further significant gains can be achieved.
\end{enumerate}

\subsection*{Performance Guarantee}

The greedy algorithm provides a performance guarantee of \( (1 - \frac{1}{e}) \) of the optimal solution for the coverage component. However, the inclusion of penalties for the number of cameras and occlusions requires careful consideration of the weights \( \lambda \) and \( \mu \) to ensure a balanced solution.

\subsection*{Conclusion}

This mathematical formulation for camera placement in 3D space using submodular optimization provides a structured approach to maximizing viewpoint coverage while minimizing occlusions and camera count. By leveraging the properties of submodular functions and employing a greedy algorithm, effective camera configurations can be derived for various applications in computer vision, surveillance, and robotics.


Submodular optimization is a rich and powerful area of combinatorial optimization that focuses on optimizing functions with the property of diminishing returns. This characteristic is particularly relevant in various applications, including machine learning, economics, game theory, and network design.

\section{Definition and Properties}
A set function \( f: 2^N \to \mathbb{R} \) defined on a finite ground set \( N \) is submodular if, for any two sets \( A \subseteq B \subseteq N \) and any element \( x \in N \setminus B \), the following inequality holds:

\[
f(A \cup \{x\}) - f(A) \geq f(B \cup \{x\}) - f(B)
\]

This property indicates that the marginal gain from adding an element \( x \) to a smaller set \( A \) is at least as great as adding it to a larger set \( B \), demonstrating diminishing returns.

%\section{Optimization Problem Formulation}
The typical optimization problem can be expressed as follows:

\[
\max_{S \subseteq N, |S| \leq k} f(S)
\]

where \( k \) is a predetermined size limit for the subset \( S \).

\section{Algorithms for Submodular Optimization}
Submodular optimization has been extensively studied, leading to various algorithms:

\begin{itemize}
	\item \textbf{Greedy Algorithms}: The classic greedy algorithm provides a \( (1 - \frac{1}{e}) \)-approximation for maximizing a non-negative submodular function under a cardinality constraint. The algorithm iteratively adds the element that provides the highest marginal gain until the size constraint is met.
	
	\item \textbf{Continuous Relaxation}: For submodular functions, continuous approaches, such as using the Lovász extension, allow for optimization over the continuous domain, often yielding stronger guarantees and results in combinatorial settings.
	
	\item \textbf{Coordinate Descent}: This technique iteratively optimizes the function by fixing all variables but one, updating that variable to improve the objective, which is particularly effective in large-scale problems.
	
	\item \textbf{Convex Optimization Methods}: When dealing with submodular functions that are also convex, standard convex optimization techniques can be employed to find optimal solutions efficiently.
\end{itemize}

\section{Applications}
The versatility of submodular optimization is evident across various domains:

\begin{itemize}
	\item \textbf{Machine Learning}: In active learning and sensor placement, submodular functions are used to model information gain, enabling more efficient data acquisition strategies.
	
	\item \textbf{Economics and Game Theory}: Submodular functions naturally model scenarios with resource allocation, where the utility of adding resources decreases as more are allocated.
	
	\item \textbf{Network Design}: In designing robust networks, submodular optimization assists in optimizing the placement of resources, such as servers or sensors, while accounting for diminishing returns in coverage or connectivity.
	
	\item \textbf{Computer Vision}: Techniques such as image segmentation and object recognition benefit from submodular functions that model the relationships between image pixels and segments.
\end{itemize}

\section{Challenges and Future Directions}
Despite its robustness, submodular optimization faces several challenges:

\begin{itemize}
	\item \textbf{Scalability}: As the size of the input set increases, algorithms may struggle with efficiency. Research into scalable algorithms that can handle large datasets is ongoing.
	
	\item \textbf{Approximation Guarantees}: While greedy algorithms provide strong theoretical guarantees, practical implementations may require further refinement to achieve optimal performance across diverse applications.
	
	\item \textbf{Generalization to Non-Submodular Functions}: Many real-world problems do not conform strictly to submodularity, prompting research into relaxation techniques or hybrid models that capture a broader range of behaviors.
	
	\item \textbf{Distributed Optimization}: In the age of big data, developing distributed algorithms that can efficiently handle submodular functions across decentralized data sources is an active area of research.
\end{itemize}

\section{Conclusion}
Submodular optimization presents a powerful framework for tackling various combinatorial optimization problems with applications across multiple fields. Its distinctive properties of diminishing returns make it suitable for modeling complex systems where resource allocation is critical. Continued advancements in algorithms and applications are essential for fully leveraging the potential of submodular optimization in addressing real-world challenges.
\section{Entropy of Random Walks on Graphs}

A random walk on a graph is a stochastic process that describes a sequence of moves through the graph's vertices (nodes). At each step, the walker moves from its current vertex to one of its adjacent vertices with a certain probability. This concept is widely used in various fields, including physics, computer science, and social network analysis.

%\section{Entropy in Information Theory}

Entropy, introduced by Claude Shannon, is a measure of uncertainty or randomness in a probability distribution. In the context of random walks, the entropy rate quantifies the average uncertainty in the walker’s position over time. A higher entropy rate indicates a more unpredictable path, while a lower entropy rate suggests a more deterministic one.

The entropy \( H(X) \) of a discrete random variable \( X \) is defined as:

\[
H(X) = -\sum_{i} p_i \log(p_i)
\]

where \( p_i \) is the probability of each outcome \( i \).

%\section{Entropy Rate of Random Walks}

The entropy rate \( h \) of a random walk can be viewed as the limit of the average entropy per step as the number of steps goes to infinity:

\[
h = \lim_{n \to \infty} \frac{1}{n} H(X_n)
\]

where \( X_n \) is the state of the random walk at time \( n \). 

For a random walk on a graph, maximizing the entropy rate can lead to more efficient exploration and sampling strategies, making it useful in applications such as network analysis, information retrieval, and optimization.

%\section{Intuition for Maximizing Entropy Rate}

Maximizing the entropy rate of a random walk involves choosing the transition probabilities between nodes to ensure a balanced and diverse exploration of the graph. Here are key intuitions behind this maximization:

\begin{itemize}
	\item \textbf{Uniformity of Access}: To achieve maximum entropy, the walker should have roughly equal access to different paths and nodes. This means avoiding biases towards certain nodes or edges, promoting a more uniform distribution of probabilities across the graph.
	
	\item \textbf{Complexity and Connectivity}: Highly connected graphs or those with complex topologies tend to yield higher entropy rates. The more pathways available, the more uncertain the walker’s next position becomes. This is particularly evident in graphs with high degrees of nodes, where multiple options for movement exist.
	
	\item \textbf{Exploration vs. Exploitation}: A strategy that maximizes the entropy rate balances exploration (visiting new or less-visited nodes) and exploitation (returning to previously visited nodes). This balance is crucial in contexts like reinforcement learning and adaptive sampling, where both novel information and reliability are desired.
\end{itemize}

%\section{Applications of Maximizing Entropy Rate}

Maximizing the entropy rate of random walks has several applications:

\begin{itemize}
	\item \textbf{Network Analysis}: Understanding the behavior of random walkers can help in analyzing the robustness and vulnerability of networks, identifying key nodes that facilitate or hinder flow.
	
	\item \textbf{Sampling Techniques}: In scenarios where samples must be drawn from a complex distribution, ensuring that random walks have high entropy can lead to more representative samples.
	
	\item \textbf{Optimization Problems}: In optimization contexts, maximizing entropy can lead to better exploration strategies, particularly in algorithms that rely on stochastic processes to navigate complex search spaces.
\end{itemize}

%\section{Conclusion}

Maximizing the entropy rate of random walks in graphs involves creating a diverse and uniform exploration strategy, allowing for efficient sampling and analysis of complex networks. Understanding the interplay between graph structure and random walk dynamics is essential for applying these concepts in practical scenarios, ranging from algorithm design to real-world network applications.


%\section{Real-Time Dynamic Scene Reconstruction}
Our objective is to develop innovative techniques for real-time 3D reconstruction and rendering, with the goal of estimating BRDF in any frequency domain, surface reflectance with specific channels, materials, enhancing the estimation of dynamic wireless channels within 3D space. This advancement will provide critical support for more accurate and efficient wireless communication systems in dynamic environments. 

In this context, we define the scene as a 3D geometry (e.g., mesh, point cloud, SDF, neural representation etc), surface parameters (e.g., BRDF, reflectance) across multiple spectral ranges, including visible light, near-infrared (NIR), infrared (IR), and terahertz frequencies. Additionally, the scene incorporates the propagation characteristics of electromagnetic waves in 3D space, e.g., considering refraction and diffraction at various frequencies.

\subsection{Paper Reading}
\begin{enumerate}
	\item NeuS: Learning Neural Implicit Surfaces by Volume Rendering for Multi-view Reconstruction, In NeurIPS, 2021. 
	\item DepthCrafter: Generating Consistent Long Depth Sequences
	for Open-world Videos.
	\item DUSt3R: Geometric 3D Vision Made Easy. In CVPR, 2024.
	\item GPS-Gaussian: Generalizable Pixel-wise 3D Gaussian Splatting for Real-time Human Novel View Synthesis. In CVPR, 2024.
	\item GigaGS: Scaling up Planar-Based 3D Gaussians for Large Scene Surface Reconstruction. 
	\item https://saghiralfasly.github.io/OSRE-Project/
\end{enumerate}


\section{Idea 1: Real-time Dynamic Scene Reconstruction via Multi-view RGBD Array}
We assume that our RGBD camera can output depth with real world unit, and we have optical flow for each frame in dynamic scene that calculates from consecutive frames by SOTA stereo matching methods, and we want to estimate each GS movement (translation) and rotation in 3D space by jointly using optical flow and depth observation.
  
%\subsubsection{Depth and Optical Flow to 3D Scene Flow}
\subsection{Estimate 3D GS Translation by Depth and Optical Flow}
Given a depth image and optical flow, along with known camera intrinsic and extrinsic parameters, we can calculate the 3D movement of points in space. The process involves converting 2D image coordinates into 3D points, then using the optical flow to estimate the 3D movement. Here are system inputs,
%\subsection{System Inputs}
\begin{itemize}
	\item Depth image \( D(u, v) \) at each pixel \((u, v)\).
	\item Optical flow \( \mathbf{\Delta u} = (\Delta u, \Delta v) \), representing the 2D displacement of pixels between consecutive frames.
	\item Camera intrinsic matrix \( K \) and extrinsic parameters (world to camera matrix) \( R \) and \( \mathbf{t} \) (rotation and translation between world and camera coordinates).
\end{itemize}

\textbf{Step 1: Depth to 3D Points in Camera Coordinates}

Given the depth \( D(u, v) \) and camera intrinsic matrix \( K \), the 2D pixel coordinates \((u, v)\) are converted to 3D coordinates \((X_c, Y_c, Z_c)\) in the camera coordinate system using:

\[
\mathbf{P}_c = D(u, v) K^{-1} \begin{pmatrix} u \\ v \\ 1 \end{pmatrix}
\]

The intrinsic matrix \( K \) is:

\[
K = \begin{pmatrix}
	f_x & 0 & c_x \\
	0 & f_y & c_y \\
	0 & 0 & 1
\end{pmatrix}
\]

So the 3D point in camera coordinates becomes:

\[
\mathbf{P}_c = D(u, v) \begin{pmatrix}
	\frac{u - c_x}{f_x} \\
	\frac{v - c_y}{f_y} \\
	1
\end{pmatrix}
\]

\textbf{Step 2: Optical Flow as 2D Disparity in the Image Plane}

Optical flow represents the 2D displacement of pixels between two frames. For a pixel at \((u, v)\), the displacement in the next frame is:

\[
\mathbf{\Delta u} = \begin{pmatrix} \Delta u \\ \Delta v \end{pmatrix} = \begin{pmatrix} u' - u \\ v' - v \end{pmatrix}
\]

Where \( (u', v') \) are the pixel coordinates in the next frame.

\textbf{Step 3: Estimate 3D Translation-(X, Y) via Optical Flow}

We can estimate the 3D position of the pixel in the next frame by using the updated pixel coordinates \((u', v')\) and assuming the depth remains approximately the same:

\[
\mathbf{P}'_c = D(u, v) K^{-1} \begin{pmatrix} u' \\ v' \\ 1 \end{pmatrix}
\]

The 3D translation \( \Delta \mathbf{P}_c = ( \Delta X_c, \Delta Y_c, \Delta Z_c ) \) in camera coordinates is:

\[
\Delta \mathbf{P}_c = \mathbf{P}'_c - \mathbf{P}_c
\]

This simplifies to:

\[
\Delta \mathbf{P}_c = D(u, v) \left( K^{-1} \begin{pmatrix} u' \\ v' \\ 1 \end{pmatrix} - K^{-1} \begin{pmatrix} u \\ v \\ 1 \end{pmatrix} \right)
\]
Here, naive $\Delta Z_c = 0$, but we can replace it as the unbiased estimation as $\Delta Z_c = D'(u', v') - D(u, v)$ to estimate the translation in z-axis in camera coordinate.
Given that the optical flow represents the change in pixel coordinates, we have:

\[
\Delta \mathbf{P}_c = D(u, v) K^{-1} \begin{pmatrix} \Delta u \\ \Delta v \\ 0 \end{pmatrix}
\]

\textbf{Step 4: Incorporate Extrinsic Parameters}

If we want to convert the 3D movement to world coordinates, we apply the camera's extrinsic parameters (rotation \( R \) and translation \( \mathbf{t} \)) to the displacement:

\[
\Delta \mathbf{P}_w = R^{-1} \Delta \mathbf{P}_c
\]

\subsection{Summary of Equations}
\begin{enumerate}
	\item \textbf{Convert 2D pixel and depth to 3D point in camera coordinates:}
	\[
	\mathbf{P}_c = D(u, v) K^{-1} \begin{pmatrix} u \\ v \\ 1 \end{pmatrix}
	\]
	\item \textbf{3D displacement from optical flow and depth:}
	\[
	\Delta \mathbf{P}_c = D(u, v) K^{-1} \begin{pmatrix} \Delta u \\ \Delta v \\ 0 \end{pmatrix}
	\]
	\item \textbf{Convert to world coordinates (if necessary):}
	\[
	\Delta \mathbf{P}_w = R^{-1} \Delta \mathbf{P}_c
	\]
\end{enumerate}


\section{Estimating the Rotation of Spherical Harmonic Function}
The rotated spherical harmonic function can be expressed as:

\[
Y_{\ell}^{m'}(\theta', \phi') = \sum_{m=-\ell}^{\ell} D^{\ell}_{m'm}(\alpha, \beta, \gamma) Y_{\ell}^{m}(\theta, \phi)
\]

where:
\begin{itemize}
	\item \( Y_{\ell}^{m}(\theta, \phi) \) is the spherical harmonic of degree \( \ell \) and order \( m \),
	\item \( \alpha, \beta, \gamma \) are the Euler angles,
	\item \( D^{\ell}_{m'm}(\alpha, \beta, \gamma) \) is the Wigner D-matrix element,
	\item \( \theta, \phi \) are the original spherical coordinates,
	\item \( \theta', \phi' \) are the spherical coordinates after the rotation.
\end{itemize}

\subsection{Wigner D-matrix}
The Wigner D-matrix \( D^{\ell}_{m'm}(\alpha, \beta, \gamma) \) is used to describe the rotation of a spherical harmonic function. It depends on the Euler angles \( \alpha \), \( \beta \), and \( \gamma \), which represent rotations about the z-axis, y-axis, and z-axis, respectively.

The general form of the Wigner D-matrix is:

\[
D^{\ell}_{m'm}(\alpha, \beta, \gamma) = e^{-im'\alpha} d^{\ell}_{m'm}(\beta) e^{-im\gamma}
\]

Where:
\begin{itemize}
	\item \( e^{-im'\alpha} \) and \( e^{-im\gamma} \) are phase factors for rotations about the z-axis.
	\item \( d^{\ell}_{m'm}(\beta) \) is the Wigner small d-matrix, which encodes the rotation about the y-axis.
\end{itemize}

The small \( d \)-matrix \( d^{\ell}_{m'm}(\beta) \) is given by:

\[
d^{\ell}_{m'm}(\beta) = \sum_{k} \frac{(-1)^{k} \sqrt{(\ell + m')!(\ell - m')!(\ell + m)!(\ell - m)!}}{(\ell + m - k)!k!(m' - m + k)!(\ell - m' - k)!} \left( \cos\frac{\beta}{2} \right)^{2\ell - 2k + m - m'} \left( \sin\frac{\beta}{2} \right)^{2k + m' - m}
\]

where the summation index \( k \) runs over all integers for which the arguments of the factorials are non-negative, and the angle \( \beta \) is the rotation angle about the y-axis.

Thus, the full Wigner D-matrix expression is:

\[
D^{\ell}_{m'm}(\alpha, \beta, \gamma) = e^{-im'\alpha} \left( \sum_{k} \frac{(-1)^{k} \sqrt{(\ell + m')!(\ell - m')!(\ell + m)!(\ell - m)!}}{(\ell + m - k)!k!(m' - m + k)!(\ell - m' - k)!} \left( \cos\frac{\beta}{2} \right)^{2\ell - 2k + m - m'} \left( \sin\frac{\beta}{2} \right)^{2k + m' - m} \right) e^{-im\gamma}
\]



\section{Estimating Rotation of GS}
To estimate the rotation of a camera or object using depth and optical flow, we can use the relationship between the 2D optical flow and the corresponding 3D motion of points, which includes both translation and rotation. Our goal is to isolate the rotational component.

%\subsection{Key Concepts}
\begin{itemize}
	\item \textbf{Optical flow} represents the movement of 2D points between frames in the image plane.
	\item \textbf{3D motion} can be decomposed into translation and rotation.
	\item The \textbf{rotation matrix} describes how points rotate in 3D space.
\end{itemize}

We use the 2D optical flow and depth information to recover the 3D rotation of the points or camera.

\textbf{Step 1: Express 3D Point Motion}

For a 3D point \( \mathbf{P}_c = (X_c, Y_c, Z_c)^T \) in the camera coordinate system, the movement between two frames is expressed as:

\[
\mathbf{P}_c' = R \mathbf{P}_c + \mathbf{t}
\]

Where:
\begin{itemize}
	\item \( R \) is the \( 3 \times 3 \) rotation matrix.
	\item \( \mathbf{t} \) is the \( 3 \times 1 \) translation vector.
	\item \( \mathbf{P}_c \) is the 3D point in the first frame.
	\item \( \mathbf{P}_c' \) is the 3D point in the second frame.
\end{itemize}

\textbf{Step 2: Project 3D Points to 2D}

The projection of a 3D point \( \mathbf{P}_c = (X_c, Y_c, Z_c)^T \) onto the 2D image plane is given by:

\[
\begin{pmatrix} u \\ v \end{pmatrix} = K \begin{pmatrix} \frac{X_c}{Z_c} \\ \frac{Y_c}{Z_c} \\ 1 \end{pmatrix}
\]

Where:
\begin{itemize}
	\item \( (u, v) \) are the 2D pixel coordinates.
	\item \( K \) is the intrinsic camera matrix.
	\item \( Z_c \) is the depth of the point in the camera coordinate system.
\end{itemize}

\textbf{Step 3: Relating Optical Flow to Rotation}

The optical flow \( \mathbf{\Delta u} = (\Delta u, \Delta v) \) between two frames is the result of the 3D motion of \( \mathbf{P}_c \). This motion can be caused by both translation and rotation.

The total 3D motion is:

\[
\mathbf{P}_c' = R \mathbf{P}_c + \mathbf{t}
\]

For rotation-only motion (i.e., assuming no translation), the relationship between the change in the 3D point due to rotation is:

\[
\mathbf{\Delta P}_c = R \mathbf{P}_c - \mathbf{P}_c = (R - I) \mathbf{P}_c
\]

Where \( I \) is the identity matrix.

\textbf{Step 4: Relate Rotation to Optical Flow}

The optical flow represents the 2D projection of the 3D motion. Thus, for each pixel \( (u, v) \), the optical flow \( \mathbf{\Delta u} = (\Delta u, \Delta v) \) is related to the 3D rotation \( R \) by:

\[
\mathbf{\Delta u} = K \frac{(R - I) \mathbf{P}_c}{Z_c}
\]

Where:
\begin{itemize}
	\item \( K \) is the camera intrinsic matrix.
	\item \( \mathbf{P}_c = (X_c, Y_c, Z_c)^T \) is the 3D point in the first frame.
\end{itemize}

\textbf{Step 5: Solving for Rotation}

Given the optical flow \( \mathbf{\Delta u} = (\Delta u, \Delta v) \), the depth \( Z_c \), and the intrinsic matrix \( K \), we can estimate the rotation matrix \( R \). The objective is to minimize the error between the observed optical flow and the predicted optical flow:

\[
\min_R \sum \left\| \mathbf{\Delta u}_{\text{observed}} - K \frac{(R - I) \mathbf{P}_c}{Z_c} \right\|^2
\]

Where \( \mathbf{\Delta u}_{\text{observed}} \) is the measured optical flow, and \( K \), \( \mathbf{P}_c \), and \( Z_c \) are known.

\textbf{Step 6: Small Angle Approximation for Rotation}

For small rotations, the rotation matrix \( R \) can be approximated using the axis-angle representation. For small rotations, \( R \) is expressed as:

\[
R \approx I + [\mathbf{\omega}]_\times
\]

Where \( \mathbf{\omega} = (\omega_x, \omega_y, \omega_z)^T \) is the rotation vector, and \( [\mathbf{\omega}]_\times \) is the skew-symmetric matrix of \( \mathbf{\omega} \):

\[
[\mathbf{\omega}]_\times = \begin{pmatrix} 
	0 & -\omega_z & \omega_y \\
	\omega_z & 0 & -\omega_x \\
	-\omega_y & \omega_x & 0 
\end{pmatrix}
\]

In this case, the 3D point movement due to rotation is:

\[
\mathbf{\Delta P}_c \approx [\mathbf{\omega}]_\times \mathbf{P}_c
\]

The optical flow is related to the rotation vector \( \mathbf{\omega} \) by:

\[
\mathbf{\Delta u} \approx K \frac{[\mathbf{\omega}]_\times \mathbf{P}_c}{Z_c}
\]

\subsection{Conclusion}

To estimate rotation using depth and optical flow:
\begin{itemize}
	\item Convert the 2D optical flow into 3D motion using the known depth and camera intrinsics.
	\item Decompose the 3D motion into translation and rotation components.
	\item Use an optimization method to estimate the rotation matrix \( R \) by minimizing the error between the observed optical flow and the predicted optical flow.
\end{itemize}

The key equation relating optical flow to rotation is:

\[
\mathbf{\Delta u} = K \frac{(R - I) \mathbf{P}_c}{Z_c}
\]

For small rotations, this simplifies to:

\[
\mathbf{\Delta u} \approx K \frac{[\mathbf{\omega}]_\times \mathbf{P}_c}{Z_c}
\]



%\section{Idea 2: Direct Dense Gaussian Point Cloud Generation by ViT}

\subsection{Point Cloud Generation Based on RGBD and Initialization of Gaussian Point Clouds}
The depth information is projected into the 3D camera coordinate system through the camera model and then transformed into the world coordinate system. Note that if the camera is single and stationary, the relationship between the camera and world coordinate systems can be ignored. However, if there are multiple cameras or a single moving camera, it is necessary to obtain the pose transformation relationship of the camera in the world coordinate system at each timestamp.

If the representation model is a Gaussian point cloud, the Radiance Field of the color component is expressed via spherical harmonics. The spherical harmonics part of the Gaussian point cloud must be initialized during the initialization stage, requiring a short training period for the entire system.

\subsection{Fast Scene Flow Estimation Based on Depth Video Stream and Optical Flow}
Scene flow is essentially the optical flow in the 3D space. We estimate the 3D motion of objects indirectly by analyzing the optical flow from 2D images and the variation in depth maps.

Let $\vec{u} = (u,v)$ be a point on the 2D image, and $\Delta u$ be the optical flow between two consecutive frames, such that $\Delta u = u_{t+1} - u_t$ and $\Delta v = v_{t+1} - v_t$.

In 3D space, a point can be represented as $\vec{p} = (x, y, z)$, and the scene flow in 3D can be expressed as $\Delta p = \vec{p}_{t+1} - \vec{p}_t$.
The 2D optical flow is, in fact, a projection of the 3D scene flow onto the camera’s image plane. Therefore, we can estimate the 3D scene flow based on the 2D optical flow and the corresponding depth map displacement.

\subsection{Fast Rotation Matrix Estimation Based on Depth Video Stream and Optical Flow}
Dynamic Gaussian point cloud flow estimation relies on scene flow to estimate the displacement of each frame of the Gaussian point cloud in 3D space. The rotation matrix is used to estimate the rotation of each Gaussian point cloud in the next frame.

\subsection{End-to-End Training of the Gaussian Point Cloud}
The entire Gaussian point cloud system is trained in an end-to-end manner, fine-tuning the system parameters to address errors in optical flow estimation and rotation estimation. This process also compensates for indirect reflection changes caused by object movement and varying lighting conditions over time.
%\section{GS for efficient Tomography Representation}
%\section{Pinhole Camera Model}

The pinhole camera model describes the projection of 3D points onto a 2D image plane. Given a 3D point \(\mathbf{P} = (X_w, Y_w, Z_w)^T\) in the world coordinate system, it is projected onto a 2D point \(\mathbf{p} = (x, y)^T\) on the camera's image plane.

\subsection{Basic Projection in Pinhole Camera Model}

In the pinhole camera model, the relationship between the 3D point in the camera coordinate system \(\mathbf{P}_c = (X_c, Y_c, Z_c)^T\) and the corresponding 2D image point \(\mathbf{p} = (x, y)\) is:

\[
\mathbf{p} = \left( \frac{X_c}{Z_c}, \frac{Y_c}{Z_c} \right)
\]

Where \(\mathbf{P}_c = (X_c, Y_c, Z_c)^T\) represents the 3D coordinates of the point in the camera coordinate system.

\subsection{Extrinsic Parameters}

The extrinsic parameters describe the transformation from the world coordinate system to the camera coordinate system. This transformation consists of a rotation matrix \(R\) and a translation vector \(\mathbf{t}\). The transformation is given by:

\[
\mathbf{P}_c = R \mathbf{P}_w + \mathbf{t}
\]

In homogeneous coordinates, this is expressed as:

\[
\begin{pmatrix}
	X_c \\
	Y_c \\
	Z_c \\
	1
\end{pmatrix}
=
\begin{pmatrix}
	R & \mathbf{t} \\
	0 & 1
\end{pmatrix}
\begin{pmatrix}
	X_w \\
	Y_w \\
	Z_w \\
	1
\end{pmatrix}
\]

where \(R\) is a \(3 \times 3\) rotation matrix and \(\mathbf{t}\) is a \(3 \times 1\) translation vector.

\subsection{Intrinsic Parameters}

The intrinsic parameters define the transformation from the camera's 3D coordinate system to the 2D image plane. These parameters are captured in the intrinsic matrix \(K\):

\[
K =
\begin{pmatrix}
	f_x & 0 & c_x \\
	0 & f_y & c_y \\
	0 & 0 & 1
\end{pmatrix}
\]

where:
\begin{itemize}
	\item \(f_x\) and \(f_y\) are the focal lengths in pixels along the \(x\) and \(y\) axes, respectively.
	\item \(c_x\) and \(c_y\) are the coordinates of the principal point in the image plane.
\end{itemize}

\subsection{Projection Equation}

To project a 3D point from world coordinates to 2D image coordinates, we use both the intrinsic and extrinsic parameters. The full projection equation is:

\[
\mathbf{p} = K [R | \mathbf{t}] \mathbf{P}_w
\]

In homogeneous coordinates, this becomes:

\[
\begin{pmatrix}
	x \\
	y \\
	1
\end{pmatrix}
=
\frac{1}{Z_c}
\begin{pmatrix}
	f_x & 0 & c_x \\
	0 & f_y & c_y \\
	0 & 0 & 1
\end{pmatrix}
\begin{pmatrix}
	R_{3 \times 3} & \mathbf{t}_{3 \times 1} \\
	0 & 1
\end{pmatrix}
\begin{pmatrix}
	X_w \\
	Y_w \\
	Z_w \\
	1
\end{pmatrix}
\]

where:
\begin{itemize}
	\item \((X_w, Y_w, Z_w)^T\) are the world coordinates of the 3D point.
	\item \((x, y)^T\) are the image plane coordinates (in pixels).
	\item \(Z_c\) is the depth of the point in the camera coordinate system.
\end{itemize}

\subsection{Complete Pinhole Camera Model}

In summary, the 3D point projection using the pinhole camera model is expressed as:

\[
\mathbf{p} = K [R | \mathbf{t}] \mathbf{P}_w
\]

where:
\begin{itemize}
	\item \(K\) is the intrinsic matrix.
	\item \(R\) and \(\mathbf{t}\) are the extrinsic parameters (rotation and translation).
	\item \(\mathbf{P}_w\) is the 3D point in world coordinates.
\end{itemize}

This equation describes the projection of a 3D point in world space to the 2D image plane using both intrinsic and extrinsic camera parameters.

%\section{Introduction}
In light of the escalating advancements in photo-realistic rendering technologies, a surge of academic interest has emerged in the fields of 3D scene acquisition, representation, and re-rendering. The seamless integration of these technologies has enabled novel applications, including live 3D streaming, real-time 3D conferencing, holographic media, augmented reality, and the metaverse, poised to revolutionize the digital landscape in the near future.
Photo-realistic rendering platforms facilitate the rendering of dynamic 3D scenes or novel viewpoints, achieved through the concurrent and precise capture of multi-view image sequences using synchronized multi-camera devices. 


1. Most of previous camera array systems adopt simple uniform sampling strategy to densely capture 3D scene, i.e., arrange cameras in a plane or hemisphere grid.
2. For the 3D scene contains occlusion and known geometry, e.g., city with many building, rooms, flat, etc. To comprehensively capture 3D information in a given scene, we need to carefully optimize camera's position and pose to cover the all the information in 3D space and reduce the hardware resource allocation, i.e., number of camera.
3. In this paper, we solve camera layout design problem that optimizes the camera pose and position to maximize the system return with a cardinality constraint, e.g., number of camera. The naive solution is discrete the valid camera pose and position, and then exhaustively search all valid configuration of camera layout, and select the maximal layout, which yields a NP-hard problems.
4. To solve the camera layout problem in a practical polynomial time manner, we observe the diminishing return property of camera layout, that is incrementally increase camera quantity in a camera layout configuration, the additional system return starts to decrease.
5. We adopt submodular optimization as weapon to solve the camera layout optimization and explore the mathematical optimal camera layout configuration. We first build a graph to represent camera arrangement configuration, each vertex is camera pose and position, and edge that connect two adjacent cameras, the weight for edge is calculated by metrics (e.g., semantic similarity, mutual coverage) of two views. We formulate camera pose and position arrangement problem as the graph selection and clustering via submodular optimization, to maximize the entropy rate of random walk in a graph.
%7. camera design problem contains the nature of diminishing return property.
%maximizes the camera maximizes the camera observation coverage and minimize the number of camera usage in the constraint of 3D occlusion and space limitation, so that optimize the quality of reconstructed 3D space. 

\subsection{introduction}
Previous camera array systems have primarily adopted uniform sampling strategies, with cameras uniformly distributed in either plane or hemisphere formations for dense 3D scene capture (1). However, when dealing with scenarios involving occlusions and well-defined geometry, such as urban landscapes with a multitude of buildings, rooms, and dwellings, optimizing camera positioning and orientation becomes crucial for capturing a comprehensive 3D dataset while minimizing hardware demands, particularly the number of cameras.

This research endeavors to address the camera layout design problem by optimizing pose and position to maximize system performance under a cardinality constraint, like the total count of cameras. The straightforward approach, though, would be to exhaustively search all possible combinations of discrete camera positions and orientations – a computationally NP-hard task (3).

We recognize the diminishing return nature of camera layouts; adding more cameras incrementally leads to a diminishing increase in system return. To tackle this problem efficiently in practical settings, we leverage submodular optimization as a powerful tool for finding near-optimal solutions within a polynomial time complexity.

To implement this, we construct a graph representation, where vertices denote individual camera placements and orientations, and edges connect adjacent cameras. Edge weights are determined by factors like semantic similarity and mutual view coverage between the corresponding views. We mathematically formulate the camera pose and position arrangement problem as a graph selection and clustering problem, utilizing submodular optimization to maximize the entropy rate of a random walk on the graph.

In doing so, we sidestep the exhaustive search by seeking a sub-optimal but computationally efficient solution that captures the diminishing returns in camera layout configuration. This approach permits us to strike a balance between coverage and resource optimization, ultimately contributing to improved 3D reconstructions in scenarios with occlusions and complex geometries.


\subsection{Others}

Previous camera array systems have predominantly relied on a straightforward uniform sampling strategy, with cameras deployed in structured formations like planar or hemispherical grids for dense 3D scene capture (1). However, in scenarios where multi-level occlusion and intricate geometry exist, such as urban landscapes with numerous buildings, rooms, and apartment complexes, an optimized camera configuration is vital for capturing comprehensive 3D data efficiently. This requires balancing the need to cover the full 3D space with minimizing hardware resources, particularly the number of cameras employed.

In this work, we delve into the intricate challenge of designing camera arrays that optimally address the dual objectives of maximizing view overlap and minimizing camera redundancy, while respecting constraints imposed by three-dimensional occlusions and spatial limitations (3). To tackle this problem, we propose a novel approach that utilizes graph theory.

By constructing a graph representation, we capture the relationship between each camera’s pose and position, where vertices are designated for individual camera configurations. Edges connect neighboring cameras, and their weights are determined by the mutual benefit between their respective viewpoints, as quantified by metrics like semantic similarity and mutual view coverage. We mathematically formulate the problem of arranging camera poses and positions as a graph selection and clustering task, leveraging the principles of submodular optimization.

Submodular optimization principles enable us to maximize the efficiency of data extraction from the graph by optimizing the entropy rate of a random walk. This not only ensures that every node in the graph contributes meaningfully to the overall coverage but also minimizes the number of cameras necessary to achieve this, ultimately enhancing the quality and resource utilization of the reconstructed 3D space. The resulting method embodies a novel, computational-driven strategy that outperforms traditional uniform sampling techniques in scenes with complex geometries and occlusions.




Modern real-time 3D acquisition systems, particularly those relying on multiple area array cameras or light field cameras, demand an extensive array of camera resources, to ensure the capture of an extensive views of scene with minimal occlusion for further 3D computation.
The redundancy of view point is necessary for enhancing 3D reconstruction accuracy and robustness, yet it also underscores the computational and resource-intensive nature of these systems that require efficient optimization strategies for real-time performance.
%These systems then rely on high-performance computing resources in the software domain to develop explicit or implicit 3D representations. 
%The process involves computationally intensive operations to optimize the unknown parameters of these models, with the ultimate aim of storing, transmitting, and rendering intricate details, including geometry, texture, lighting, and radiometry, contributing to the immersive experiences these technologies provide.


%While numerous past research efforts in enhancing the overall performance of 3D acquisition and representation systems have adhered to a “more is better” mindset, where, on the hardware front, increased camera resources are thought to facilitate denser viewpoint sampling, and in the software realm, larger parameter sets are employed for more intricate model expressions. Concomitantly, the system’s soft and hardware configuration and deployment may exhibit “less restraints” in certain circumstances. Firstly, redundant resource allocation exists, indicating that judiciously reducing resources does not significantly degrade the system’s performance. Secondly, it is important to note that simply scaling up the hardware and software resources does not lead to unlimited performance gains, as there is a plateau. Thirdly, optimizing critical parameters has a pronounced impact on enhancing system performance. We observe that the 3D acquisition and expression systems exhibit a characteristic diminishing marginal returns: as more resources are invested, the incremental performance gains gradually decline, and targeted optimization of key configuration parameters can trigger substantial leaps in overall performance.

While numerous past research efforts have predominantly focused on increasing in hardware resources (e.g., more cameras) and software parameters, a more nuanced understanding is emerging. These efforts have indicated that while more resources initially contribute to greater precision, the system’s performance improvement plateaus beyond a certain point due to redundancy and resource constraints. This suggests that optimizing key parameters holds critical importance, as it can lead to significant leaps in system efficiency, despite the diminishing returns trend.


However, the existing theoretical framework lacks comprehensive support for optimizing the intricate mathematical link between discrete hardware-software configuration parameters in 3D information processing systems. The lack of methodological guidance hampers the development of effective loss functions and optimization strategies, which are fundamental for designing optimal systems. Discrete parameter optimizations, often NP-hard, call for exhaustive search methods, making them impractical in real-world scenarios.


Consequently, the ambitions of achieving resource efficiency and targeted performance enhancements, epitomized by “cost-effective” and “targeted resource allocation,” pose challenges in the practical adoption of 3D-enhanced applications, such as holographic projections and virtual reality, to adaptable and scalable platforms. This lack of tailored optimization prevents smooth integration and systematic improvements in performance, leaving the full potential of these systems largely unexplored.


To address this issue, adopting a system-centric approach to model and optimize discrete parameters in 3D acquisition and expression is of paramount significance. Establishing a mathematical model between discrete parameters and system performance offers a foundation for quantitative design and optimization, enabling systematic efficiency improvements across soft and hardware platforms and facilitating efficient configuration under resource constraints.


In the context of diminishing returns, the inherent submodularity can be strategically leveraged to optimize these systems, using polynomial-time algorithms that can navigate intractable optimization challenges associated with NP-hard problems. These algorithms yield superior approximation techniques, conducive to optimizing multiple discrete parameters simultaneously.


Lastly, the discrete nature of system parameter combinations aligns closely with practical challenges encountered in industry and everyday applications, particularly in large-scale systems with complex parameters lacking clear mathematical relationships. Optimizing these discrete parameters directly allows for precise, theoretical calculations of limits and optimal configurations, delivering superior design outcomes for large-scale projects, thereby highlighting the pivotal role of discrete parameter optimization in unlocking untapped potential.

%With the increasing prevalence of high-speed, low-latency internet connectivity and photo-realistic rendering technologies, a surging research interest has been observed in the fields of 3D scene acquisition, representation and re-rendering, make live 3D streaming, real-time 3D conferencing, holographic media, augmented reality, and the metaverse become emerging applications in the near future.

%With the development of photo-realistic rendering technologies, the re-rendering of dynamic 3D scene or novel view can be achieved by densely and simultaneously capturing multi-view image sequences by exposure-synchronized multi-camera system.  At present, the prevailing real-time 3D information acquisition systems, e.g., camera array, light field camera, etc, for realistic scenes necessitate extensive camera resources on the hardware side to gather multi-angle, unobstructed, and redundant-angled colored or depth data. Subsequently, in the software domain, explicit or implicit 3D models are constructed, leveraging high-performance computing devices for computationally intensive operations. These operations involve optimizing unknown parameters for the 3D models, with the ultimate goal of storing, transmitting, and rendering details such as structure, texture, lighting, and radiometry.

%Over an extended period, research efforts in enhancing the overall performance of 3D acquisition and representation systems have adhered to a “more is better” mindset, where, on the hardware front, increased camera resources are thought to facilitate denser viewpoint sampling, and in the software realm, larger parameter sets are employed for more intricate model expressions. Concomitantly, the system’s soft and hardware configuration and deployment may exhibit “less restraints” in certain circumstances. Firstly, redundant resource allocation exists, indicating that judiciously reducing resources does not significantly degrade the system’s performance. Secondly, it is important to note that simply scaling up the hardware and software resources does not lead to unlimited performance gains, as there is a plateau. Thirdly, optimizing critical parameters has a pronounced impact on enhancing system performance. We observe that the 3D acquisition and expression systems exhibit a characteristic diminishing marginal returns: as more resources are invested, the incremental performance gains gradually decline, and targeted optimization of key configuration parameters can trigger substantial leaps in overall performance.

%In academic writing:



%However, in current scientific theories, there is a lack of solid theoretical ground and optimization techniques specifically addressing the mathematical relationship between discrete hardware and software configuration parameters and the performance of 3D information acquisition and expression systems. First, there is an absence of clear methodological guidance for designing loss functions and optimization strategies, which are fundamental components. Second, optimizing general discrete set parameters often confronts non-continuous NP-hard problems, where finding the global optimum involves exhaustive enumeration of all possible parameter combinations, rendering it impractical in real-world scenarios. As a result, the objectives of “efficient resource allocation” and “maximizing effectiveness” during system deployment become challenging, hindering the deployment of applications such as holographic video and virtual reality in flexible and scalable environments. These applications struggle to be deployed with targeted performance upgrades and optimizations, and their performance ceilings remain elusive.

%In academic terms:

%Past research efforts have, predominantly, pursued a “quantity over quality” strategy in enhancing the overall system performance of 3D acquisition and rendering, assuming that augmenting hardware with additional camera resources and software with larger parameter sets leads to greater detail and precision. Meanwhile, there is a possibility, inherent in the system, that resource allocation displays redundancy: although minimizing resources can impact performance minimally. Simultaneously, unyielding linear performance gains are not guaranteed by mere increases in hardware and software capacity, as it appears to operate within an upper bound. It is evident that the systems exhibit a marginal decrement in benefits: the incremental performance gain diminishes with continued investment of resources, and targeted optimization of crucial parameters can bring about significant leaps in system efficiency. This empirical observation underscores the importance of rigorous optimization strategies in these systems to optimize key metrics and counteract diminishing returns.

%Nonetheless, the existing theoretical framework falls short in offering explicit support for optimizing the intricate mathematical relationship between those discrete hardware-software configurational parameters critical to 3D information processing systems. Firstly, the lack of methodological guidance obstructs the development of effective loss functions and optimization strategies, both of which are paramount to the system’s design. Second, common optimizations of discrete parameter sets are inherently non-convex NP-hard problems, necessitating an exhaustive search for the global optimum, rendering such an approach infeasible in practical implementations.

%Consequently, the ambitious goals of resource-efficient “bang for the buck” and targeted performance enhancement using “precise resource allocation” become barriers in the real-world deployment of 3D information-enhanced applications, like holographic projections and virtual reality. These applications face difficulties integrating seamlessly into adaptable and scalable platforms, preventing systematic and targeted improvements in their performance. As such, the fundamental limits of these systems remain largely uncharted territory.

%As such, addressing the mathematical modeling and combinatorial optimization of discrete set parameters in 3D information acquisition and expression systems with a systems approach holds significant theoretical and practical implications. To begin with, a mathematical model of the general relationship between discrete parameters and system performance offers theoretical guidance for quantitative design and optimization of such systems. This allows for systematic efficiency improvements in both software and hardware, and enables the calculation of optimal system configurations under limited resources, thereby delimiting the system’s potential performance limits.

%Furthermore, in the context of diminishing marginal returns, the inherent submodularity of the system can be exploited for efficient optimization. By employing algorithms with optimal system performance growth in polynomial time complexity, one can navigate the intractable complexities of NP-hard problems typically inherent to such optimizations. These methods provide excellent approximation ratios, facilitating the joint optimization of multiple discrete parameters.

%Lastly, the discrete nature of system parameter combinations inherently stands out from continuous system optimization methodologies. Discrete optimization problems align more closely with the practical challenges encountered in daily life and industry, especially inlarge-scale systems where complex parameters lack clear mathematical correlations. By directly optimizing the discrete parameters, it is possible to compute both theoretical limits and optimal solutions without sacrificing the accuracy of problem representation. This approach yields superior design solutions and optimal system configurations, proving invaluable for the meticulous optimization of large-scale projects, underscoring the pivotal role of discrete parameter optimization.
%
	
\section*{Entropy Rate of Spherical Harmonics on a Surface}

\subsection*{1. Understanding Spherical Harmonics}

Spherical harmonics \( Y_l^m(\theta, \phi) \) are a set of orthogonal functions defined on the surface of a sphere. They are used in various fields, including physics and engineering, to solve problems with spherical symmetry. The general form is:
\[
Y_l^m(\theta, \phi) = \sqrt{\frac{(2l+1)}{4\pi} \frac{(l-m)!}{(l+m)!}} P_l^m(\cos \theta) e^{im\phi}
\]
where \( P_l^m \) are the associated Legendre polynomials, \( \theta \) is the polar angle, and \( \phi \) is the azimuthal angle.

\subsection*{2. Entropy Rate}

Entropy rate is a measure of the uncertainty per unit time in a stochastic process. For a continuous stochastic process \( X(t) \), the entropy rate \( H(X) \) can be defined as:
\[
H(X) = \lim_{T \to \infty} \frac{1}{T} H(X_{[0,T]})
\]
where \( H(X_{[0,T]}) \) is the differential entropy of the process over the interval \([0, T]\).

\subsection*{3. Stochastic Process on a Sphere}

To apply this concept to spherical harmonics on a surface, we need to consider a stochastic process defined on the spherical surface. Suppose \( Z(\theta, \phi) \) is a random field on the sphere that can be expanded in terms of spherical harmonics:
\[
Z(\theta, \phi) = \sum_{l=0}^{\infty} \sum_{m=-l}^{l} a_l^m Y_l^m(\theta, \phi)
\]
where \( a_l^m \) are random coefficients.

\subsection*{4. Calculating the Entropy Rate}

The entropy rate of \( Z(\theta, \phi) \) can be approached by analyzing the entropy rate of the coefficients \( a_l^m \). If we assume the coefficients \( a_l^m \) are independent and identically distributed (i.i.d.) random variables with some known distribution, the entropy rate can be computed from the distribution of these coefficients.

\paragraph*{1. Distribution of \( a_l^m \):}
If \( a_l^m \) follows a normal distribution \( \mathcal{N}(0, \sigma^2) \), the entropy \( H(a_l^m) \) for each coefficient is given by:
\[
H(a_l^m) = \frac{1}{2} \log (2\pi e \sigma^2)
\]

\paragraph*{2. Summing the Entropies:}
Since there are \((2l+1)\) coefficients for each \( l \), the total entropy for a given \( l \) is:
\[
H_l = (2l + 1) \cdot \frac{1}{2} \log (2\pi e \sigma^2)
\]

\paragraph*{3. Entropy Rate:}
If we consider the total entropy up to a certain \( l_{\text{max}} \), the entropy rate \( H(Z) \) can be approximated by:
\[
H(Z) = \lim_{l_{\text{max}} \to \infty} \frac{1}{4\pi} \sum_{l=0}^{l_{\text{max}}} (2l + 1) \cdot \frac{1}{2} \log (2\pi e \sigma^2)
\]
Simplifying, we get:
\[
H(Z) = \frac{1}{4\pi} \sum_{l=0}^{\infty} (2l + 1) \cdot \frac{1}{2} \log (2\pi e \sigma^2)
\]


	
\section*{Spherical Harmonic Functions}

Spherical harmonic functions are a set of orthogonal functions defined on the surface of a sphere. They play a crucial role in various areas of physics, mathematics, and engineering, particularly in solving problems with spherical symmetry, such as in quantum mechanics, geophysics, and computer graphics.

\subsection*{Definition}

A spherical harmonic function \( Y_l^m(\theta, \phi) \) is defined by two integers:
\begin{itemize}
	\item \( l \): the degree, which is a non-negative integer ( \( l = 0, 1, 2, \ldots \) ).
	\item \( m \): the order, which is an integer such that \( -l \leq m \leq l \).
\end{itemize}

The general form of a spherical harmonic is given by:
\[
Y_l^m(\theta, \phi) = \sqrt{\frac{(2l+1)}{4\pi} \frac{(l-m)!}{(l+m)!}} P_l^m(\cos \theta) e^{im\phi}
\]

where:
\begin{itemize}
	\item \( \theta \) is the polar angle (colatitude) with \( 0 \leq \theta \leq \pi \).
	\item \( \phi \) is the azimuthal angle (longitude) with \( 0 \leq \phi \leq 2\pi \).
	\item \( P_l^m \) are the associated Legendre polynomials.
\end{itemize}

\subsection*{Associated Legendre Polynomials}

The associated Legendre polynomials \( P_l^m(x) \) are defined for \( -1 \leq x \leq 1 \) and are related to the Legendre polynomials \( P_l(x) \) by:
\[
P_l^m(x) = (1 - x^2)^{\frac{m}{2}} \frac{d^m}{dx^m} P_l(x)
\]

The Legendre polynomials \( P_l(x) \) themselves are solutions to the Legendre differential equation:
\[
\frac{d}{dx} \left( (1 - x^2) \frac{d P_l(x)}{dx} \right) + l(l+1) P_l(x) = 0
\]

\subsection*{Orthogonality}

Spherical harmonics are orthogonal functions on the surface of a sphere. The orthogonality condition is expressed as:
\[
\int_0^{2\pi} \int_0^{\pi} Y_l^m(\theta, \phi) Y_{l'}^{m'}(\theta, \phi)^* \sin \theta \, d\theta \, d\phi = \delta_{ll'} \delta_{mm'}
\]
where \( \delta_{ll'} \) and \( \delta_{mm'} \) are the Kronecker delta functions, and \( Y_{l'}^{m'}(\theta, \phi)^* \) is the complex conjugate of \( Y_{l'}^{m'}(\theta, \phi) \).

\subsection*{Properties}

\begin{enumerate}
	\item \textbf{Normalization:}
	Spherical harmonics are normalized such that:
	\[
	\int_0^{2\pi} \int_0^{\pi} |Y_l^m(\theta, \phi)|^2 \sin \theta \, d\theta \, d\phi = 1
	\]
	
	\item \textbf{Completeness:}
	Any square-integrable function on the sphere can be expanded in terms of spherical harmonics. This is known as spherical harmonic expansion:
	\[
	f(\theta, \phi) = \sum_{l=0}^{\infty} \sum_{m=-l}^{l} a_l^m Y_l^m(\theta, \phi)
	\]
	where \( a_l^m \) are the spherical harmonic coefficients given by:
	\[
	a_l^m = \int_0^{2\pi} \int_0^{\pi} f(\theta, \phi) Y_l^m(\theta, \phi)^* \sin \theta \, d\theta \, d\phi
	\]
	
	\item \textbf{Symmetry:}
	Spherical harmonics exhibit symmetry properties such as:
	\begin{itemize}
		\item Conjugate symmetry: \( Y_l^{-m}(\theta, \phi) = (-1)^m Y_l^m(\theta, \phi)^* \)
		\item Parity: \( Y_l^m(\pi - \theta, \phi + \pi) = (-1)^l Y_l^m(\theta, \phi) \)
	\end{itemize}
\end{enumerate}

\subsection*{Applications}

\begin{itemize}
	\item \textbf{Quantum Mechanics:} Used in solving the Schrödinger equation for atoms and molecules, particularly for the angular part of wavefunctions.
	\item \textbf{Geophysics:} Applied in representing the Earth's gravitational and magnetic fields.
	\item \textbf{Computer Graphics:} Utilized in lighting calculations and in the representation of spherical data.
\end{itemize}

In summary, spherical harmonics are fundamental in the analysis and representation of functions defined on the sphere, providing a powerful tool in both theoretical and applied sciences.



%\section{Preliminary Knowledge of Spherical Harmonic Functions}

Spherical harmonic functions are essential in various scientific and engineering disciplines, providing a way to describe functions on the surface of a sphere. Here is some preliminary knowledge to help understand spherical harmonic functions.

\subsection*{Basic Concepts}

\paragraph{Spherical Coordinates:}
Spherical coordinates are a system for describing points in three-dimensional space using three parameters:
\begin{itemize}
	\item \textbf{Radius (r)}: The distance from the origin to the point.
	\item \textbf{Polar Angle (θ)}: The angle between the positive z-axis and the line connecting the origin to the point. It ranges from 0 to π.
	\item \textbf{Azimuthal Angle (φ)}: The angle in the x-y plane from the positive x-axis. It ranges from 0 to 2π.
\end{itemize}

\paragraph{Functions on a Sphere:}
Functions defined on the surface of a sphere depend only on the angles θ and φ, not on the radius. These functions can be expanded in terms of spherical harmonics.

\subsection*{Spherical Harmonic Functions}

\paragraph{Definition:}
Spherical harmonic functions \( Y_l^m(\theta, \phi) \) are solutions to the angular part of Laplace's equation in spherical coordinates. They are defined as:
\[
Y_l^m(\theta, \phi) = \sqrt{\frac{(2l+1)}{4\pi} \frac{(l-m)!}{(l+m)!}} P_l^m(\cos \theta) e^{im\phi}
\]
where:
\begin{itemize}
	\item \( l \) is the degree (non-negative integer).
	\item \( m \) is the order (integer such that \( -l \leq m \leq l \)).
	\item \( P_l^m \) are the associated Legendre polynomials.
\end{itemize}

\paragraph{Associated Legendre Polynomials:}
These polynomials \( P_l^m(x) \) are derived from the Legendre polynomials \( P_l(x) \) and are given by:
\[
P_l^m(x) = (1 - x^2)^{\frac{m}{2}} \frac{d^m}{dx^m} P_l(x)
\]
where \( P_l(x) \) solves the Legendre differential equation:
\[
\frac{d}{dx} \left( (1 - x^2) \frac{d P_l(x)}{dx} \right) + l(l+1) P_l(x) = 0
\]

\subsection*{Properties}

\paragraph{Orthogonality:}
Spherical harmonics are orthogonal functions, satisfying:
\[
\int_0^{2\pi} \int_0^{\pi} Y_l^m(\theta, \phi) Y_{l'}^{m'}(\theta, \phi)^* \sin \theta \, d\theta \, d\phi = \delta_{ll'} \delta_{mm'}
\]
where \( \delta_{ll'} \) and \( \delta_{mm'} \) are Kronecker delta functions.

\paragraph{Normalization:}
They are normalized so that:
\[
\int_0^{2\pi} \int_0^{\pi} |Y_l^m(\theta, \phi)|^2 \sin \theta \, d\theta \, d\phi = 1
\]

\paragraph{Completeness:}
Any square-integrable function on the sphere can be expanded as a sum of spherical harmonics:
\[
f(\theta, \phi) = \sum_{l=0}^{\infty} \sum_{m=-l}^{l} a_l^m Y_l^m(\theta, \phi)
\]
where the coefficients \( a_l^m \) are given by:
\[
a_l^m = \int_0^{2\pi} \int_0^{\pi} f(\theta, \phi) Y_l^m(\theta, \phi)^* \sin \theta \, d\theta \, d\phi
\]

\subsection*{Applications}

\begin{itemize}
	\item \textbf{Quantum Mechanics:} Used to describe the angular part of the wavefunctions of particles in spherical potentials, such as electrons in atoms.
	\item \textbf{Geophysics:} Applied in representing the Earth's gravitational and magnetic fields.
	\item \textbf{Computer Graphics:} Utilized in techniques such as spherical harmonics lighting to efficiently simulate the way light interacts with surfaces.
\end{itemize}

\subsection*{Summary}

Spherical harmonic functions \( Y_l^m(\theta, \phi) \) provide a powerful tool for representing and analyzing functions on the surface of a sphere. They are characterized by their degree \( l \) and order \( m \), and are defined in terms of associated Legendre polynomials and complex exponentials. Their orthogonality and completeness properties make them invaluable in various fields of science and engineering.


\section*{Representing 3D Geometry Using Spherical Harmonic Functions}

Spherical harmonic functions are widely used to represent 3D geometries, especially when these geometries have spherical or nearly spherical symmetry. This representation is particularly useful in fields such as computer graphics, medical imaging, and geophysics.

\subsection*{Steps to Represent 3D Geometry}

The key idea is to represent the shape of a 3D object as a function on the unit sphere. This function can then be expanded into a series of spherical harmonics. Here's an outline of the process:

\begin{enumerate}
	\item \textbf{Parameterize the Surface:} Parameterize the surface of the 3D object using spherical coordinates \((\theta, \phi)\).
	\item \textbf{Function on the Sphere:} Define a function \( f(\theta, \phi) \) that represents the radius of the surface from the origin as a function of the spherical coordinates. For example, \( f(\theta, \phi) \) could be the distance from the origin to the surface at the given angles.
	\item \textbf{Spherical Harmonic Expansion:} Expand the function \( f(\theta, \phi) \) in terms of spherical harmonics:
	\[
	f(\theta, \phi) = \sum_{l=0}^{\infty} \sum_{m=-l}^{l} a_l^m Y_l^m(\theta, \phi)
	\]
	where \( Y_l^m(\theta, \phi) \) are the spherical harmonics, and \( a_l^m \) are the coefficients.
	\item \textbf{Reconstruction:} Use the spherical harmonic coefficients to reconstruct the surface geometry.
\end{enumerate}

\subsection*{Example: Representing a Simple 3D Geometry}

Below is a Python script to represent a simple 3D geometry using spherical harmonics. For demonstration, we'll represent a deformed sphere (like an ellipsoid).

	

%\input{overview}
%\section{Method}
The design of camera arrays is a crucial problem in computer vision, robotics, and multimedia applications. The primary objective is to optimize the pose (orientation) and position of multiple cameras in a system to maximize coverage of viewpoints while minimizing the total number of cameras used and reducing occlusions. This complex challenge involves various interrelated factors, including geometry, camera models, valid configuration setup, sensor characteristics, and application-specific requirements.

In a typical camera array design problem, several key objectives are established. (1) maximizing scene coverage is essential; the camera array should effectively cover a predefined 3D space, ensuring that every point of interest is within the field of view (FOV) of at least one camera. This objective is crucial for enabling view synthesis and 3D reconstruction of entire scene without losing any points in views. (2) the overall design should aim to minimize the number of cameras while achieving maximum coverage. This approach not only reduces costs and complexity of camera array's configuration, but also mitigates the challenges associated with data management and processing. (3) The arrangement of cameras must be optimized to reduce occlusions—instances in which one object obstructs the view of another. Strategic positioning can enhance visibility and detail capture, particularly in dynamic scenes.

Designing an effective camera array necessitates consideration of several critical factors.
First, the field of view (FOV) is fundamental; each camera possesses a specific FOV, typically determined by its lens characteristics. A thorough understanding of these parameters is essential for calculating the coverage provided by each camera and assessing the degree of overlap among them.
Second, the spatial configuration of cameras significantly influences the overall performance of the array. The design must take into account the physical environment, including the height and layout of the area to be monitored, as well as potential obstacles that may obstruct views.
Third, the camera pose—encompassing its orientation and tilt—plays a crucial role in determining the perspective from which scenes are captured. Optimizing the pose is vital for maximizing coverage and minimizing blind spots.
Lastly, data fusion is essential; integrating data from multiple cameras can yield a more comprehensive view of the monitored area. The design must address how to efficiently combine data from various angles and perspectives to enhance overall quality and information content.

\subsection{Problem Formulation}
In a camera array system consisting of \( N \) cameras, each with \( M \) possible poses, we seek to select \( K \) cameras to maximize the view coverage of a predefined three-dimensional space. This problem can be formally described as follows:

Given a set of cameras \( C = \{c_1, c_2, \ldots, c_N\} \) and a corresponding set of poses for each camera \( P = \{p_{ij} : 1 \leq j \leq M\} \), where \( p_{ij} \) denotes the \( j \)-th pose of the \( i \)-th camera, the objective is to select a subset \( S \subseteq C \) with \( |S| = K \) such that the total view coverage of the selected cameras is maximized.

The coverage function \( f(S) \), which quantifies the total volume of the 3D space covered by the selected camera configurations, exhibits submodularity. This property implies that the incremental gain in coverage from adding an additional camera diminishes as more cameras are included in the selection. Formally, for any subsets \( A \subseteq B \subseteq C \) and any camera \( c \in C \setminus B \):

\[
f(A \cup \{c\}) - f(A) \geq f(B \cup \{c\}) - f(B)
\]

To efficiently solve this optimization problem, we can employ submodular optimization techniques, which allow for the identification of an approximately optimal configuration. This approach leverages the properties of submodular functions to guarantee that a greedy algorithm can yield a solution within a provable bound of the optimal coverage. Specifically, the greedy selection process iteratively adds the camera that provides the maximum marginal increase in the coverage function until \( K \) cameras have been selected.

Thus, the problem of selecting \( K \) cameras from \( N \) candidates, each with \( M \) poses, to maximize the view coverage of a known 3D space can be effectively addressed using submodular optimization methods, leading to an efficient and scalable solution for camera array configuration.


\subsection{Optimization Methods}
Several optimization techniques can be employed to solve this problem:

\begin{itemize}
	\item \textbf{Greedy Algorithms}: A greedy approach can iteratively select camera positions that offer the greatest increase in coverage while checking for redundancy and occlusions.
	
	\item \textbf{Genetic Algorithms}: Evolutionary strategies can explore a broader search space, combining and mutating camera configurations to evolve optimal placements over generations.
	
	\item \textbf{Simulated Annealing}: This probabilistic technique can escape local optima by allowing less optimal configurations at early stages, refining toward better solutions as iterations progress.
	
	\item \textbf{Integer Linear Programming (ILP)}: The problem can be formulated as an ILP, where the decision variables represent camera placements, and the objective function encapsulates the coverage maximization and occlusion minimization criteria.
\end{itemize}

The camera array design problem has numerous applications across various fields:
\textbf{Surveillance Systems}: In security and surveillance, optimizing camera placement is crucial for effective monitoring of large areas while minimizing equipment costs.
\textbf{Autonomous Vehicles}: Camera arrays in autonomous systems must provide comprehensive environmental coverage to ensure safe navigation and obstacle detection.
\textbf{Robotics}: In robotic perception, camera arrays can enhance the ability to perceive the environment, facilitating better decision-making and interaction with objects.
\textbf{Virtual Reality (VR) and Augmented Reality (AR)}: Optimally designed camera arrays can improve the quality of immersive experiences by ensuring full environmental coverage from multiple viewpoints.

The camera array design problem is a complex optimization challenge that necessitates a careful balance between maximizing coverage, minimizing camera count, and reducing occlusions. By employing advanced mathematical and algorithmic techniques, designers can create efficient camera systems tailored to specific applications, leading to enhanced performance and data quality. As technology advances, the exploration of innovative solutions for this problem will continue to evolve, driving progress in computer vision and related fields.


\section{Mathematical Formulation for Camera Placement in 3D Space via Submodular Optimization}

In the context of camera placement for maximizing view coverage in a three-dimensional space, we formulate the problem as follows:

Let \( C = \{c_1, c_2, \ldots, c_N\} \) represent a set of \( N \) cameras, where each camera \( c_i \) has a finite set of poses \( P_i = \{p_{i1}, p_{i2}, \ldots, p_{iM}\} \). The goal is to select a subset \( S \subseteq C \) of \( K \) cameras, each with an optimal pose, to maximize the coverage of a predefined 3D space \( V \).

The coverage function \( f(S) \) is defined as:

\[
f(S) = \text{Coverage}(S, V)
\]

where \( \text{Coverage}(S, V) \) quantifies the volume of the 3D space \( V \) that is visible from the selected cameras in \( S \).

The objective is to solve the following optimization problem:

\[
\max_{S \subseteq C, |S| = K} f(S)
\]

%\section{Submodularity of the Coverage Function}

The coverage function \( f(S) \) is submodular if it satisfies the diminishing returns property:

\[
f(A \cup \{c\}) - f(A) \geq f(B \cup \{c\}) - f(B) \quad \forall A \subseteq B \subseteq C, \, c \in C \setminus B
\]

This property implies that the incremental gain in coverage decreases as more cameras are added to the selection.

%\section{Greedy Algorithm for Optimization}

To find an approximate solution, we can implement a greedy algorithm as follows:

\begin{enumerate}
	\item Initialize \( S = \emptyset \).
	\item While \( |S| < K \):
	\begin{enumerate}
		\item Select the camera \( c^* \) that maximizes the marginal increase in coverage:
		
		\[
		c^* = \arg\max_{c \in C \setminus S} \left( f(S \cup \{c\}) - f(S) \right)
		\]
		
		\item Add \( c^* \) to the set \( S \).
	\end{enumerate}
	\item Return the selected set \( S \) and their corresponding optimal poses.
\end{enumerate}

%\section{Conclusion}

The mathematical formulation of the camera placement problem in 3D space via submodular optimization provides a structured approach to maximize view coverage. By leveraging the properties of submodular functions, we can utilize efficient algorithms to achieve near-optimal solutions, thereby enhancing the effectiveness of camera array configurations in various applications.



%\input{exp}
%\input{ext_exp}
%\input{conclusion}
%\bibliographystyle{IEEEtran}
%\bibliography{trans}
%{\small
%	\bibliographystyle{ieee_fullname}
%	\bibliography{trans.bib}
%}
\vfill

\end{document}


